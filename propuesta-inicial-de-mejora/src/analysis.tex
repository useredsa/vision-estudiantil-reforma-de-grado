\chapter{Comentarios sobre el título actual}\label{chap:analysis}

% Comentar que ya existe un documento de crítica
% elaborado por la junta de facultad

En la Junta de Facultad del 29 de mayo de 2019
se estableció la creación de
la Comisión de Análisis del Grado en Ingeniería Informática,
que tenía como papel
identificar los problemas existentes en
el plan de estudios y su enfoque metodológico,
determinar si era necesario actualizar los contenidos del grado y
hacer las proposiciones de mejora que considerase oportunas.
El resultado fue un informe que identificaba los siguientes problemas,
llamados de primer o segundo nivel,
según requieran modificaciones mayores al plan de estudios o no.
A continuación enumeramos los problemas que se detallan en ese informe.

\begin{enumerate}
    \item Problemas de \emph{primer nivel}.
    \begin{enumerate}
        \item Es necesario actualizar los contenidos
        debido a los avances experimentados en la disciplina.
        \item\label{anl:red-opt}
        Es necesario reducir la optatividad de las menciones de cara a
        cumplir con la normativa de la Universidad de Murcia.
        \item\label{anl:opt-in-3}
        Es conveniente situar algunas competencias obligatorias en cuarto,
        con el fin de aprovechar la madurez del estudiante.
        \item Es necesario modificar la estructura del título
        para llevar a cabo los \cref{anl:red-opt,anl:opt-in-3}.
        \item Es necesario reforzar los fundamentos matemáticos de la carrera,
        para facilitar a las asignaturas avanzadas desarrollar su docencia.
    \end{enumerate}

    \item Problemas de \emph{segundo nivel}.
    \begin{enumerate}
        \item Reducir el detalle en la descripción de las materias.
        \item Lograr mayor continuidad en el desarrollo de los contenidos
        a lo largo del grado.
        \item Fomentar el conocimiento básico
        frente al conocimiento tecnológico.
        \item Mejorar en competencias en resolución de problemas.
        \item Coordinación en el uso de lenguajes de programación.
        \item Definir claramente las competencias transversales.
        \item Otras consideraciones.
    \end{enumerate}
\end{enumerate}

En el enfoque y la propuesta para el nuevo título,
intentaremos abordar cada punto y aportar soluciones efectivas.

% TODO 4 abril
\section{Comentarios del estudiantado al plan actual}
A continuación recogemos el sentimiento de algunos estudiantes y exestudiantes 
al ser preguntados sobre el grado en general, asignaturas específicas o posibles
mejoras.

\begin{quote}{Anónimo}
    No tiene sentido saberse la normativa entera del cmmi.
\end{quote}

\begin{quote}{Anónimo}
    Que sea un requerimiento llevar los proyectos con git.
\end{quote}

\begin{quote}{Anónimo}
    La teoría de POO es la documentación de java.
\end{quote}

\begin{quote}{Anónimo}
    Lo que yo diría es que no creo que sea bueno
    no ver nada de redes neuronales ni machine learning antes de la rama,
    pero 4 asignaturas de redes donde los primeros temas son iguales en las 4.
\end{quote}

\begin{quote}{Anónimo}
    Yo creo que a partir de cierto momento,
    tipo sobre nuestro tercero a lo mejor,
    se debería dar importancia tipo pues que
    si una función puede fallar aunque no suela hacerlo
    halla una comprobación o un try,catch,finally en algún sitio apropiado y tal,
    y sobre todo que se den ciertas directrices sobre como hacerlo o donde.
    En plan, que ya se hace pero que no estaría de mas que
    se explicara en algún momento como decidir que hacer en caso de error
    para recuperarse y tal.
\end{quote}

\begin{quote}{Sempere}
    Menos algoritmos en papel y más proyectos reales.
\end{quote}

\begin{quote}{Sempere}
    Una asignatura en general de TIC y Cloud,
    donde se hable de seguridad informática y de código.
\end{quote}

\begin{quote}{Anónimo}
    En la mención de computación no se aprende nada.
\end{quote}

\begin{quote}{Anónimo}
    El contenido del grado está totalmente desactualizado.
\end{quote}

\begin{quote}{Anónimo}
    El grado está bien, tocas todo, pero se da mucho demasiado general y
    muchas cosas no se utilizan, a lo mejor faltaría especializarse antes
    y mejor.
\end{quote}

\begin{quote}{Anónimo}
    En primero, lógica es una asignatura que el concepto general puede que 
    sea necesario, pero está muy mal enfocada. Además se sale de primero 
    sin saber programar decentemente.
\end{quote}

\begin{quote}{Antonio Mangas}
    Quizás el grado debería acercarse más a las empresas, ver que buscan y 
    profundizar en esas cosas. 
\end{quote}

\begin{quote}{Marcos Domínguez}
    En general, el grado está bien, salvo el tridente GPDS/TDS/PDS que dejan
    mucho que desear. En comparación con otros compañeros de máster de otras
    universidades, salimos bien entrenados. Además el resto de mis actuales
    compañeros no han visto contenido de computadores o de redes, 
    contenido que veo necesario en una ingeniería informática,sus grados
    están muy enfocados en el desarrollo de software y en eso se nota que llevan
    muchas horas de experiencia. Aunque no estaría mal
    quitar alguna asignatura de computadores en favor de software, ya que 
    es lo que actualmente manda en el mercado laboral.
\end{quote}

\begin{quote}{Marcos Domínguez}
    GPDS/TDS/PDS se pueden organizar mucho mejor. GPDS y PDS tienen mucha carga 
    de trabajo y van después de TDS, error.
\end{quote}


% TODO 4 abril
\section{Evaluación de la metodología actual}

Con respecto a la metodología que se ve actualmente en el grado, se intenta
seguir el mismo esquema en todas las asignaturas del grado, a pesar de que
estas presentan características diferentes. Es decir, hay asignaturas
eminentemente prácticas que no requieren de una relación de teoría/prácticas
donde se tiene una proporción 55/45 respectivamente sobre 100 horas lectivas, 
que es el actual reparto.

Por otro lado, quizás se deberían ver ciertos contenidos básicos de forma
autónoma por parte del alumno. Un ejemplo de esto es cuando se ve un nuevo
lenguaje de programación y se gasta tiempo en ver entornos y hacer una 
familiarización con el lenguaje. Estos primeros pasos podrían acelerarse
gracias a la programación de pequeños programas ayudados por jueces online y
tutoriales. Esta idea se inspira en la página que ofrece de apoyo el profesor %TODO*******************
Laguna, a través de la página
%*********************** meter pagina de laguna
. Que ha tenido una gran acogida entre el estudiantado y valoran positivamente.
Esto permitiría a los profesores poder profundizar en contenidos más complejos.

Además, permite al estudiante no depender tanto de la cátedra del profesor y evitaría
comparaciones entre grupos con profesores diferentes y partir de una base común.