\chapter{Comentarios sobre el título actual}\label{chap:analysis}

% Comentar que ya existe un documento de crítica
% elaborado por la junta de facultad

En la Junta de Facultad del 29 de mayo de 2019
se estableció la creación de
la Comisión de Análisis del Grado en Ingeniería Informática,
que tenía como papel
identificar los problemas existentes en
el plan de estudios y su enfoque metodológico,
determinar si era necesario actualizar los contenidos del grado y
hacer las proposiciones de mejora que considerase oportunas.
El resultado fue un informe~\cite{analysis-comission}
que identificaba los siguientes problemas,
llamados de primer o segundo nivel,
según requieran modificaciones mayores al plan de estudios o no.
A continuación citamos los problemas que se detallan en ese informe.

\begin{enumerate}
    \item Problemas de \emph{primer nivel}.
    \begin{enumerate}
        \item Es necesario actualizar los contenidos
        debido a los avances experimentados en la disciplina.
        \item\label{anl:red-opt}
        Es necesario reducir la optatividad de las menciones de cara a
        cumplir con la normativa de la Universidad de Murcia.
        \item\label{anl:opt-in-3}
        Es conveniente situar algunas competencias obligatorias en cuarto,
        con el fin de aprovechar la madurez del estudiante.
        \item Es necesario modificar la estructura del título
        para llevar a cabo los \cref{anl:red-opt,anl:opt-in-3}.
        \item Es necesario reforzar los fundamentos matemáticos de la carrera,
        para facilitar a las asignaturas avanzadas desarrollar su docencia.
    \end{enumerate}

    \item Problemas de \emph{segundo nivel}.
    \begin{enumerate}
        \item Reducir el detalle en la descripción de las materias.
        \item Lograr mayor continuidad en el desarrollo de los contenidos
        a lo largo del grado.
        \item Fomentar el conocimiento básico
        frente al conocimiento tecnológico.
        \item Mejorar en competencias en resolución de problemas.
        \item Coordinación en el uso de lenguajes de programación.
        \item Definir claramente las competencias transversales.
        \item Otras consideraciones.
    \end{enumerate}
\end{enumerate}

En el enfoque y la propuesta para el nuevo título,
intentaremos abordar cada punto y aportar soluciones efectivas.

% TODO 4 abril
\section{Comentarios del estudiantado al plan actual}

\textbf{Editorial:} Esta sección se encuentra a medias.
De momento solo incluye algunos comentarios que hemos podido recoger.
En el futuro, intentaremos acompañar los comentarios de reflexiones.

A continuación recogemos el sentimiento de algunos estudiantes y egresados 
al ser preguntados sobre
el grado en general, asignaturas específicas o posibles mejoras.

\subsection{Comentarios globales}

\begin{quote}{Sempere}\label{qte:less-algorithms-in-paper}
    Menos algoritmos en papel y más proyectos reales.
\end{quote}

\begin{quote}{Anónimo}\label{qte:cmmi}
    [Como ejemplo de qué se evalua en el grado]
    No tiene sentido saberse la normativa entera del cmmi.
\end{quote}

\begin{quote}{Anónimo}
    Que sea un requerimiento llevar los proyectos con git.
\end{quote}

\begin{quote}{Ángela Martínez}
    La mayoría de los profesores explican bien.
\end{quote}

\begin{quote}{Anónimo}
    Lo que yo diría es que no creo que sea bueno
    no ver nada de redes neuronales ni machine learning antes de la rama,
    pero sí 4 asignaturas de redes donde los primeros temas
    son iguales en las 4.
\end{quote}

\begin{quote}{Anónimo}
    En la mención de computación no se aprende nada.
\end{quote}

\begin{quote}{Anónimo}
    El contenido del grado está totalmente desactualizado.
\end{quote}

\begin{quote}{Anónimo}
    El grado está bien, tocas todo,
    pero muchos contenidos de abordan de manera demasiado general y
    después no se utilizan.
    A lo mejor faltaría especializarse antes y mejor.
\end{quote}

\begin{quote}{Antonio Mangas}
    Tras haber finalizado la carrera,
    considero que quizás el grado debería acercarse más a las empresas,
    ver qué buscan y profundizar en esos puntos. 
\end{quote}

\begin{quote}{Marcos Domínguez}\label{qte:marcos-contents}
    En general, el grado está bien,
    salvo el tridente GPDS/TDS/PDS que dejan mucho que desear.
    En comparación con otros compañeros de máster de otras universidades,
    salimos bien entrenados.
    Además el resto de mis actuales compañeros no han visto
    contenido de computadores o de redes, 
    contenido que veo necesario en una ingeniería informática.
    Sus grados están muy enfocados en el desarrollo de software.
    Y en eso se nota que llevan muchas horas de experiencia.

    Aunque no estaría mal quitar alguna asignatura de computadores
    en favor de software,
    ya que es lo que actualmente manda en el mercado laboral.
\end{quote}

\subsection{Comentarios particulares}

\begin{quote}{Anónimo}
    La teoría de POO es la documentación de java.
\end{quote}

\begin{quote}{Anónimo}
    Yo creo que a partir de cierto momento,
    tipo sobre nuestro tercero a lo mejor,
    se debería dar importancia al manejo de errores.
    Que si una función puede fallar aunque no suela hacerlo
    haya una comprobación o un \lstinline{try - catch - finally}
    en algún sitio apropiado.
    Pero sobre todo, que se den ciertas directrices sobre cómo hacerlo y dónde,
    más que dejar pasar todas las excepciones.
    Que aunque en cierta medida ya se hace, no estaría de más que
    se explicara en algún momento cómo decidir qué hacer en caso de error.
\end{quote}

\begin{quote}{Marcos Domínguez}
    GPDS/TDS/PDS se pueden organizar mucho mejor.
    GPDS y PDS tienen mucha carga de trabajo y van después de TDS.
    Error.
\end{quote}

\begin{quote}{Ángela Martínez}
    GPDS no aporta nada. DPII es un poco floja, pero al menos 
    aprendes a redactar un proyecto y a trabajar con más personas.
\end{quote}

\begin{quote}{Ángela Martínez}
    Muchas asignaturas están descompensadas.
    Por ejemplo, en PDS no da tiempo a realizar las prácticas,
    en Compresión Multimedia parece que quieren que les escribamos un TFG,
    mientras que PCD dudo que lleguen a ser 3 créditos reales. 
\end{quote}

\subsection{Comentarios con sugerencias}

\begin{quote}{Sempere}
    Una asignatura en general de TIC y Cloud,
    donde se hable de seguridad informática y de código.
\end{quote}

\begin{quote}{Anónimo}
    En primero, lógica es una asignatura que el concepto general puede que 
    sea necesario, pero está muy mal enfocada. Además se sale de primero 
    sin saber programar decentemente.
\end{quote}

\section{Evaluación de la metodología actual}

En esta sección vamos a comentar la actual metodología
que se aplica en algunas asignaturas y que podría ser
cambiada por otros hábitos. Muchos exestudiantes no ven 
carencias significativas en el temario del grado con 
respecto a compañeros de profesión egresados de otras 
facultades, pero si un problema en el método con el que 
se enseñan ciertas habilidades transversales a cualquier 
tipo de informático.

Son numerosas las asignaturas del grado actual que 
se detienen en ver de manera teórica muchos de los
algoritmos que se pueden considerar básicos o no tanto.
Donde, lejos de probar la habilidad del estudiante programando
estos algoritmos delante de un ordenador, son partidarios
de escribirlo con papel y bolígrafo. Esta práctica carece 
de sentido, puesto que es delante de un ordenador y con 
un problema de aplicación donde el estudiante puede mostrar 
más claramente que es poseedor del conocimiento para programar
el algoritmo, mientras que delante de un folio, se puede perder 
claridad y confundir, no solo al estudiante, si no al profesor que 
corrija ese examen.

En los comentarios de alumnos, son numerosos que se quejan de 
la metodología utilizada en GPDS y PDS, asignaturas que, lejos
de brindar metodologías para fomentar los buenos hábitos, se han 
convertido en asignaturas con gran contenido teórico que los
estudiantes memorizan y no llegan a aplicar en el marco de un
proyecto.


\subsection{Adaptación de los horarios a la metodología}

Actualmente, independientemente de la metodología que sería apropiada para 
cada asignatura, estas estan adaptadas a una modalidad con dos horas de
teoría y una hora y cuarenta minutos de práctica semanales. Esto provoca
graves desequilibrios en aquellas que no tienen suficiente contenido en
alguna de estas partes. Una solución puede ser liberar la rigidez de este 
esquema a clasificar las asignaturas en tres categorías: aquellas que 
cumplen con las horas actualmente, aquellas que requieren de más teoría,
y finalmente aquellas que requieren un mayor peso de las prácticas. 

\subsection{Metodología autónoma}

Por otro lado, quizás se deberían ver ciertos contenidos básicos de forma
autónoma por parte del alumno. Un ejemplo de esto es cuando se ve un nuevo
lenguaje de programación y se gasta tiempo en ver entornos y hacer una 
familiarización con el lenguaje. Estos primeros pasos podrían acelerarse
gracias a la programación de pequeños programas ayudados por jueces online y
tutoriales. Esta idea se inspira en la página que ofrece de apoyo el profesor 
Laguna, a través de la página \url{https://ants.inf.um.es/staff/jlaguna/tp/}
, que ha tenido una gran acogida entre el estudiantado y valoran positivamente.
Esto permitiría a los profesores poder profundizar en contenidos más complejos.

Además, permite al estudiante no depender tanto de la cátedra del profesor y 
evitaría comparaciones entre grupos con profesores diferentes y partir de 
una base común.
