\chapter{Comentarios sobre el título actual}\label{chap:analysis}

% Comentar que ya existe un documento de crítica
% elaborado por la junta de facultad

En la Junta de Facultad del 29 de mayo de 2019
se estableció la creación de
la Comisión de Análisis del Grado en Ingeniería Informática,
que tenía como papel
identificar los problemas existentes en
el plan de estudios y su enfoque metodológico,
determinar si era necesario actualizar los contenidos del grado y
hacer las proposiciones de mejora que considerase oportunas.
El resultado fue un informe~\cite{analysis-comission}
que identificaba los siguientes problemas,
llamados de primer o segundo nivel,
según requieran modificaciones mayores al plan de estudios o no.
A continuación citamos los problemas que se detallan en ese informe.

\begin{enumerate}
    \item Problemas de \emph{primer nivel}.
    \begin{enumerate}
        \item Es necesario actualizar los contenidos
        debido a los avances experimentados en la disciplina.
        \item\label{anl:red-opt}
        Es necesario reducir la optatividad de las menciones de cara a
        cumplir con la normativa de la Universidad de Murcia.
        \item\label{anl:opt-in-3}
        Es conveniente situar algunas competencias obligatorias en cuarto,
        con el fin de aprovechar la madurez del estudiante.
        \item \label{anl:mod-structure} Es necesario
        modificar la estructura del título para llevar a cabo los
        \cref{anl:red-opt,anl:opt-in-3}.
        \item Es necesario reforzar los fundamentos matemáticos de la carrera,
        para facilitar a las asignaturas avanzadas desarrollar su docencia.
    \end{enumerate}

    \item Problemas de \emph{segundo nivel}.
    \begin{enumerate}
        \item Reducir el detalle en la descripción de las materias.
        \item Lograr mayor continuidad en el desarrollo de los contenidos
        a lo largo del grado.
        \item Fomentar el conocimiento básico
        frente al conocimiento tecnológico.
        \item \label{anl:problem-solving} Mejorar en
        competencias en resolución de problemas.
        \item Coordinación en el uso de lenguajes de programación.
        \item Definir claramente las competencias transversales.
        \item Otras consideraciones.
    \end{enumerate}
\end{enumerate}

En el enfoque y la propuesta para el nuevo título,
intentaremos abordar cada punto y aportar soluciones efectivas.

% TODO 4 abril
\section{Comentarios del estudiantado al plan actual}

\textbf{Editorial:} Esta sección se encuentra a medias.
De momento solo incluye algunos comentarios que hemos podido recoger.
En el futuro, intentaremos acompañar los comentarios de reflexiones.

A continuación recogemos el sentimiento de algunos estudiantes y egresados
al ser preguntados sobre
el grado en general, asignaturas específicas o posibles mejoras.

\subsection{Comentarios globales}

\begin{quote}{Sempere}\label{qte:less-algorithms-in-paper}
    Menos algoritmos en papel y más proyectos reales.
\end{quote}

\begin{quote}{Anónimo}\label{qte:cmmi}
    [Como ejemplo de qué se evalua en el grado]
    No tiene sentido saberse la normativa entera del cmmi.
\end{quote}

\begin{quote}{Anónimo}
    Que sea un requerimiento llevar los proyectos con git.
\end{quote}

\begin{quote}{Ángela Martínez}
    La mayoría de los profesores explican bien.
\end{quote}

\begin{quote}{Anónimo}
    Lo que yo diría es que no creo que sea bueno
    no ver nada de redes neuronales ni machine learning antes de la rama,
    pero sí 4 asignaturas de redes donde los primeros temas
    son iguales en las 4.
\end{quote}

\begin{quote}{Anónimo}
    En la mención de computación no se aprende nada.
\end{quote}

\begin{quote}{Anónimo}
    El contenido del grado está totalmente desactualizado.
\end{quote}

\begin{quote}{Anónimo}
    El grado está bien, tocas todo,
    pero muchos contenidos de abordan de manera demasiado general y
    después no se utilizan.
    A lo mejor faltaría especializarse antes y mejor.
\end{quote}

\begin{quote}{Antonio Mangas}
    Tras haber finalizado la carrera,
    considero que quizás el grado debería acercarse más a las empresas,
    ver qué buscan y profundizar en esos puntos.
\end{quote}

\begin{quote}{Marcos Domínguez}\label{qte:marcos-contents}
    En general, el grado está bien,
    salvo el tridente GPDS/TDS/PDS que dejan mucho que desear.
    En comparación con otros compañeros de máster de otras universidades,
    salimos bien entrenados.
    Además el resto de mis actuales compañeros no han visto
    contenido de computadores o de redes,
    contenido que veo necesario en una ingeniería informática.
    Sus grados están muy enfocados en el desarrollo de software.
    Y en eso se nota que llevan muchas horas de experiencia.

    Aunque no estaría mal quitar alguna asignatura de computadores
    en favor de software,
    ya que es lo que actualmente manda en el mercado laboral.
\end{quote}

\subsection{Comentarios particulares}

\begin{quote}{Anónimo}
    La teoría de POO es la documentación de java.
\end{quote}

\begin{quote}{Anónimo}
    Yo creo que a partir de cierto momento,
    tipo sobre nuestro tercero a lo mejor,
    se debería dar importancia al manejo de errores.
    Que si una función puede fallar aunque no suela hacerlo
    haya una comprobación o un \lstinline{try - catch - finally}
    en algún sitio apropiado.
    Pero sobre todo, que se den ciertas directrices sobre cómo hacerlo y dónde,
    más que dejar pasar todas las excepciones.
    Que aunque en cierta medida ya se hace, no estaría de más que
    se explicara en algún momento cómo decidir qué hacer en caso de error.
\end{quote}

\begin{quote}{Marcos Domínguez}
    GPDS/TDS/PDS se pueden organizar mucho mejor.
    GPDS y PDS tienen mucha carga de trabajo y van después de TDS.
    Error.
\end{quote}

\begin{quote}{Ángela Martínez}
    GPDS no aporta nada. DPII es un poco floja, pero al menos
    aprendes a redactar un proyecto y a trabajar con más personas.
\end{quote}

\begin{quote}{Ángela Martínez}
    Muchas asignaturas están descompensadas.
    Por ejemplo, en PDS no da tiempo a realizar las prácticas,
    en Compresión Multimedia parece que quieren que les escribamos un TFG,
    mientras que PCD dudo que lleguen a ser 3 créditos reales.
\end{quote}

\subsection{Comentarios con sugerencias}

\begin{quote}{Sempere}
    Una asignatura en general de TIC y Cloud,
    donde se hable de seguridad informática y de código.
\end{quote}

\begin{quote}{Anónimo}
    En primero, lógica es una asignatura que el concepto general puede que
    sea necesario, pero está muy mal enfocada. Además se sale de primero
    sin saber programar decentemente.
\end{quote}

\section{Evaluación de la metodología actual}

% Motivo por el que hay que hablar de la metodología
% y si se puede tratar en una reforma de grado

Cuando se hace una reforma de grado
los cambios no deberían centrarse solo en los contenidos.
En este caso, el motivo principal por el que es necesario un proceso de reforma
en vez pequeñas modificaciones es para cumplir el \cref{anl:mod-structure}.
Sin embargo, a raíz de los comentarios de los estudiantes podemos ver
que algunas de las quejas no están relacionadas con los contenidos,
de hecho algunos comentarios (\cref{qte:marcos-contents})
hablan positivamente de los contenidos en comparación a otra universidades.
Muchas de las quejas se centran o bien en asignaturas concretas
o en la metodología docente, como el \cref{qte:less-algorithms-in-paper}.
Nuestro objetivo en esta sección es hacer una crítica de la metodología actual.

\subsection{Algoritmos en papel y bolígrafo}

Volviendo sobre el \cref{qte:less-algorithms-in-paper},
son numerosas las asignaturas del grado actual que
evalúan a sus estudiantes haciéndoles reproducir
la ejecución de un algoritmo en papel.
Es más, no es que se estudie de forma teórica los algoritmos
(corrección y complejidad vistas mediante demostraciones),
ni que se enseñe a diseñar algoritmos similares\footnote{
    Una práctica habitual para aprender a diseñar algoritmos
    o para demostrar que se comprenden es hacer pequeñas modificaciones
    sobre los algoritmos que se estudian para conseguir que hagan otras cosas.
}.
La corrección del algoritmo se supone cierta o
justificada por funcionar con una amplia cantidad de casos.
El trabajo que se pide al estudiante es saber simular el algoritmo.
Por ende, la preparación del examen se convierte en un entrenamiento
para ganar la experiencia que permite
reproducir el algoritmo a mano sin cometer errores.
Una actividad que requiere tiempo y aporta pocos beneficios
más allá de obtener una buena calificación.

Saber reproducir el algoritmo en papel
no significa saber implementarlo correctamente.
Por ejemplo, en papel es habitual trabajar con una coleción pequeña,
porque los casos no pueden ser muy grandes si se resuelven a mano,
y representarla de manera extensiva (escribiendo sus elementos).
Si el algoritmo requiere en un punto
escoger el elemento más pequeño de la colección y borrarlo,
el estudiante simplemente lo tacha.
Sin embargo, a la hora de implementar el algoritmo,
las diferentes formas de representar esa colección cambian
la complejidad de las operaciones.
O incluso qué operaciones están disponibles.
Es decir, hay detalles de implementación que no se tratan sobre el papel.

Desde el Área Estudiantil nos gustaría hacer énfasis en este problema.
Si se considera que hay que conocer un algoritmo,
entonces debería estudiarse de manera integral,
demostrando su corrección y su complejidad,
e implementándolo en lugar de reproduciéndolo en papel.

% \subsection{Competencias en resolución de problemas}

% En el apartado anterior comentábamos que muchos alumnos están en contra
% de que la enseñanza se base en la reproducción de algoritmos en papel.
% Creemos que cambiar esa metodología es un primer paso para
% mejorar en competencias de resolución de problemas,
% uno de los objetivos de la comisión (\cref{anl:problem-solving}).

% El paso que desde el Área Estudiantil consideramos complicado es
% integrar las matemáticas en las asignaturas de informática.

\subsection{La evaluación teórica de habilidades de gestión}

% Exceso de generalidad/contenido teórico,
% en especial en asignaturas de gestión, que son más prácticas

En los comentarios de estudiantes,
se recogen muchas quejas en contra de asignaturas como
\subject{Gestión de Proyectos de Desarrollo Software (GPDS)} y
\subject{Procesos de Desarrollo de Software (PDS)}.
Nosotros queremos hacer énfasis en el \cref{qte:cmmi},
porque es posible que dé en el clavo del problema.

Capability Maturity Model Integration (CMMi) es un modelo para mejorar
los procesos de desarrollo de software.
Sin duda, conocer un modelo de gestión de procesos software
es importante para el ingeniero informático que se encamina al mundo laboral.
Pero, ¿es necesario aprender punto por punto la normativa?
Eso es lo que cuestiona ese comentario anónimo.

En cierta medida, es como si para enseñar Git,
un sistema de control de versiones que los alumnos piden aprender,
se decidiese que se va a evaluar a los alumnos preguntándoles
por los comandos que aparecen en el manual,
cuando en realidad, los desarrolladores utilizan
unos pocos comandos en su día a día y
consultan ese manual en caso de duda.
Cuando los estudiantes piden aprender Git,
probablemente buscan ir aplicándolo en sus proyectos de la carrera
porque eso les beneficia,
y aprenderlo de manera práctica en el proceso.

El argumento es que a la hora de aprender algunas cosas,
como usar una herramienta,
prima más el aprendizaje práctico que el teórico.
Y sobre todo, que no debe evaluarse el aprendizaje con un examen memorístico.

Aun así, cuando es necesario evaluar una herramienta en un examen,
hay mejores opciones.
En la asignatura del grado actual
\subject{Introducción a los Sistemas Operativos}
se hace un examen práctico con acceso a los manuales
en el que el tiempo está restringido de manera que
no se pueden consultar a cada momento,
pero tampoco exige conocer todas las opciones de cada comando de memoria.

Es posible que por eso, GPDS y PDS se hayan convertido para muchos
en asignaturas que;
lejos de brindar metodologías para fomentar los buenos hábitos
y aplicarlas a los proyectos en los que se implica el alumno cada cuatrimestre
hasta hacerlas una práctica común,
sin necesitar de una evaluación teórica;
fuerzan a los estudiantes a memorizar y estudiar herramientas fuera de contexto.

\subsection{Adaptación de los horarios}

Actualmente,
las asignaturas del Grado en Ingeniería Informática
están adaptadas a una modalidad con dos horas de
teoría y una hora y cuarenta minutos de práctica semanales,
independientemente de la distribución que sería apropiada para cada asignatura.
Esto provoca graves desequilibrios en aquellas que
no tienen suficiente contenido en alguna de estas partes.
Una solución puede ser liberar la rigidez de este
esquema a clasificar las asignaturas en tres categorías:
aquellas que cumplen con las horas actualmente,
aquellas que requieren de más teoría,
y finalmente aquellas que requieren un mayor peso de las prácticas.

Esto permitiría que asignaturas como
\subject{Introducción a la Programación} de primer curso
pudiesen tener más horas dedicadas a la práctica,
algo que pedimos en nuestra propuesta.

% \subsection{Metodología autónoma}

% Por otro lado, quizás se deberían ver ciertos contenidos básicos de forma
% autónoma por parte del estudiante.
% Un ejemplo de esto es cuando se ve un nuevo lenguaje de programación y
% se gasta tiempo en ver entornos y hacer una familiarización con el lenguaje.
% Estos primeros pasos podrían acelerarse gracias a
% la programación de pequeños programas ayudados por jueces online y tutoriales.
% Esta idea se inspira en la página que ofrece de apoyo el profesor Laguna,
% a través de la página \url{https://ants.inf.um.es/staff/jlaguna/tp/},
% que ha tenido una gran acogida entre el estudiantado y valoran positivamente.
% Esto permitiría a los profesores poder profundizar en contenidos más complejos.

% Además, permite al estudiante no depender tanto de la cátedra del profesor y
% evitaría comparaciones entre grupos con profesores diferentes y partir de
% una base común.
