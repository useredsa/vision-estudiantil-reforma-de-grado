\chapter{Enfoque de la reforma de grado}\label{chap:approach}

\section{Introducción}

% Párrafo de introducción.
% Se pueden dar muchos enfoques a una reforma de grado.
% Este capítulo muestra el de la comisión de estudiantes.

Una reforma de grado es un proceso burocrático que implica tanto a los
profesores de cada una de las áreas como a los estudiantes de la titulación.
Con un equipo tan grande,
es importante comunicar los objetivos individuales para
tener una meta común clara.
En este capítulo damos a conocer el enfoque del área estudiantil de la comisión.

% Descripción del capítulo.

En primer lugar, hablaremos de nuestros objetivos.
Después esclareceemos nuestras opiniones respecto a
las tecnologías que deben enseñarse en el grado y
cómo debería manejarse la optatividad.
En el resto del capítulo,
haremos una descripción de
los contenidos que queremos que tenga cada curso del grado.

% Objetivos del capítulo.
% Comunicar nuestra opinión al resto de la comisión.
% Justificar nuestra propuesta de grado.

El primer objetivo del capítulo es
dar a conocer al resto de la comisión nuestra opinión
para poder después integrarla con la suya.
El segundo es mostrar el enfoque que
hay detrás de las reformas que se describen en el \cref{chap:proposal}.

% TODO 4 abril
\section{Objetivos}

% TODO 4 abril
\section{Tecnologías}

Una de las cuestiones que queremos abordar es
la elección de tecnologías en el grado.
A lo largo del texto utilizaremos la palabra \emph{tecnología} para referirnos a
las herramientas que utilizan los ingenieros informáticos en su trabajo.
Ejemplos de tecnologías son
los lenguajes de programación,
como C, Java o Python;
librerías y frameworks de código,
que normalmente van asociadas a un lenguaje de programación;
productos como simuladores, depuradores y sistemas gestores de bases de datos; y
otras herramientas como sistemas de control de versiones.
Una tecnología tiene una aplicación de cara a un problema.
Por tanto, podemos agruparlas según su propósito.

Para un ingeniero informático,
conocer las tecnologías es un asunto de vital importancia,
especialmente en el mundo laboral,
donde dominar unas u otras abre la puerta a diferentes empleos.
En su formación,
un ingeniero informático aprende a utilizar varias tecnologías
y en las universidades recae la tarea de decidir cuáles serán.

El día a día muestra que
una vez que se ha utilizado una tecnología que resuelve un problema,
es mucho más sencillo aprender a utilizar otra que resuelva el mismo problema.
Por tanto, es común que los profesores argumenten que
no es tan relevante qué tecnologías se enseñe en el grado,
que el énfasis debe estar en aportar una enseñanza atemporal
que permita al estudiante encontrar trabajo durante toda su vida laboral,
independientemente de que las tecnologías predominantes cambien en el futuro.
Al contrario que los profesores,
los estudiantes suelen tener interés en aprender las tecnologías que
les aporten más oportunidades en el mundo laboral actual.

Desde el punto de vista lógico,
si elegir una tecnología u otra no es determinante
en las habilidades que se adquieren,
teniendo en cuenta que aprender las nuevas ofrece más oportunidades laborales,
es estrictamente mejor aprender las nuevas.
Sin embargo, cambiar la tecnología utilizada
supone actualizar todo el material,
mejorado cuatrimestre a cuatrimestre,
de una asignatura,
lo cual es una tarea costosa para la universidad.

Reconociendo que la actualización de las tecnologías es necesaria,
la Comisión de Análisis del Grado en Ingeniería Informática
determinó que era necesario
reducir el nivel de especificación de las tecnologías en las guías docentes,
ya que impide que el grado pueda adaptarse a los cambios en el mundo laboral.
El Área Estudiantil de
la Comisión de Reforma del Grado en Ingeniería Informática
también defiende que
el grado debe centrarse en enseñar esas habilidades básicas,
que no dependen de la tencología que se use,
pero que es esencial que se mantenga actualizado respecto al mundo laboral actual.
Y que, por tanto, es necesario actualizar algunas de las asignaturas del grado.
Además de que es importante elegir, de cara a los próximos años,
las tecnologías que se enseñarán y unificarlas para que
sean las mismas en diferentes asignaturas.
También debe considerarse en la comisión la opción de hacer el grado más flexible,
permitiendo a los estudiantes elegir qué tecnologías utilizar
(en especial en los últimos cursos).

% Mención a la atemporalidad
% Es decir, cómo se va a manejar en las guías docentes
% para que las tecnologías puedan ir adaptándose.

\section{Optatividad}

Uno de los principales problemas que tiene que abordar
la Comisión de Reforma
es cómo reducir la optatividad del grado. %TODO (\cref{sec:optativity-problem}).
Además, la comisión también valora mover parte de la optatividad a tercer curso.

Desde el Área Estudiantil creemos que, efectivamente,
es razonable mover algunas de las asignaturas de las menciones a tercero para
incrementar el período de especialización
(dando mayor experiencia en el área al finalizar el grado)
y poder hacer espacio para
incorporar algunas asignaturas obligatorias en cuarto,
como solicitan algunos docentes.

En cuanto a la reducción en la oferta de optatividad,
creemos que es muy difícil mantener las $5$ menciones actuales
con solo $25$ asignaturas optativas (en lugar de $40$).
Nos gustaría apoyar la idea de crear una oferta abierta de asignaturas
y permitir a los estudiantes elaborar su propio currículum.
Como punto a favor, creemos que
sería más fácil ofrecer $25$ asignaturas a elegir
que intentar que las menciones compartan asignaturas.
Como punto en contra, habría que decidir
cómo se cumpliría con la oferta de cupos,
lo cual se podría organizar para que
se asignase asignaturas a los estudiantes según su media;
y cómo se distribuiría el horario,
pues los estudiantes no podrían asistir a dos asignaturas que se solapasen.
En conjunto, creemos que es una opción que hay que estudiar.
Sin embargo, otras alternativas pueden también pueden funcionar bien.

\section{Primer curso}

El primer y segundo curso son normalmente la base de un grado,
porque son los cursos donde se instruye al estudiante en ramas que desconoce y
en las que se profundizará más adelante.

En el área estudiantil creemos que primero es un curso poco aprovechado
porque no se trabaja suficiente materia.
En el títulno actual, una persona que finalice primero
no tiene suficiente experiencia programando.
Por ende, hace más difíciles asignaturas de segundo curso
que requieren esa experiencia, como
\subject{Algoritmos y Estructuras de Datos},
\subject{Programación Orientada a Objetos} o
\subject{Programación Concurrente y Distribuida}.

Creemos que el primer curso del grado debe centrarse en
enseñar a los estudiantes una buena base de
programación,
herramientas informáticas y
matemáticas.
Creemos que invertir más créditos de primero en estas categorías
dotará a los estudiantes de mejor base y
mejorará su rendimiento durante el resto del grado. 
Además, estas categorías son también importantes en otras carreras,
como el resto de Ingenierías, Matemáticas o Física.
Por tanto, también dotan al estudiante de una buena base
en el caso de que decida cambiar de carrera.

\subsection{Competencias en programación}

Creemos que el principal objetivo del primer curso debe ser
dar al estudiante una gran experiencia programando.
Para cumplirlo, la mayor parte del tiempo debe dedicarse,
precisamente, a programar.
Desde el área estudiantil diríamos que
en torno un \SI{50}{\percent} del tiempo.
El resultado esperado es que la experiencia que acumulen los estudiantes
les permita razonar en las asignaturas que verán más adelante,
como \subject{Programación Orientada a Objetos} o
\subject{Algoritmos y Estructuras de Datos}\footnotemark.

\footnotetext{O aquellas que enseñen esos contenidos en el nuevo plan.}

El nivel adquirido tras
haber superado satisfactoriamente las asignaturas de primer curso
debería ser suficiente para implementar ideas complejas en
un lenguaje de programación iterativo
interactuando con la entrada y la salida estándar.
Esto podría ser un resultado del aprendizaje que habría que definir,
aunque en este punto de la carrera,
no se exigiría que las soluciones que encontrasen fuesen las mejores en
complejidad, estructura y prácticas de programación
(como limpieza de código).
El \cref{tab:first-course-programming-examples}
muestra una serie de problemas que deberíamos conseguir que
un estudiante de primero pudiese implementar,
junto con una estimación del tiempo necesario para hacerlo.
Dos tercios del cuadro podrían abordarse en
una asignatura de programación durante el primer cuatrimestre.

\begin{table}[h]
    \centering
    \begin{tabular}{ll}
        \textbf{Tiempo} & \textbf{Problema} \\
        \hline
        \multirow{3}{*}{Hasta \SI{30}{min}.}
        & Escribir un número en otra base. \\
        & Hallar la transpuesta de una matriz. \\
        & Sumar fracciones. \\
        \hline
        \multirow{4}{*}{De \SIrange{30}{50}{min}.}
        & Factorizar un número. \\ 
        & Ordenar una secuencia de números. \\
        & Encontrar las ocurrencias de una cadena en un crucigrama. \\
        & Comprobar si un cuadrado $9\times 9$ es solución de un sudoku. \\
        \hline
        \multirow{3}{*}{De \SIrange{1}{3}{h}.}
        & Determinar si un elemento se encuentra en un array ordenado. \\
        & Hallar la unión de dos arrays ordenados. \\
        & Evaluar s-expresiones de la forma \lstinline!* 4 + 1 2!.
    \end{tabular}
    \caption{
        Competencias mínimas en programación tras superar el primer curso.
        Para cada problema, se incluye el tiempo esperado para resolverlo.
    }
    \label{tab:first-course-programming-examples}
\end{table}

\subsection{Matemáticas}

El otro área en el que hay que enfatizar en primero son las matemáticas.
La Arquitectura de Computadores, la Ciberseguridad y la Computación
son áreas de la Ingeniería Informática donde
las matemáticas son imprescindibles.
El resto de áreas, como la Ingeniería del Software,
también necesitan mentes resolutivas y con buena capacidad de abstracción.
De hecho, uno de los problemas que
identificó la Comisión de Análisis del Grado
fue que los estudiantes no tenían una buena base matemática.

De igual manera,
las matemáticas también son la base del resto de ingenierías.
Concentrar mayor parte de las asignaturas de matemáticas en el primer curso
también facilitaría a los estudiantes cambiar de carrera
aprovechando parte de los créditos aprobados.

Las materias que se imparten en primero del grado actualmente son
\subject{Álgebra y Matemática Discreta},
\subject{Fundamentos Lógicos de la Informática}\footnotemark,
\subject{Cálculo} y
\subject{Estadística}.
Por un lado, nuestra opinión es que
la cantidad de créditos de matemáticas en primero es adecuada.
Sin embargo, creemos que la asignatura de lógica
podría ser sustituida por otra asignatura de matemáticas.
Además, la asignatura \subject{Álgebra y Matemática Discreta}
podría dividirse en dos asignaturas,
una de Álgebra Lineal y otra que tratase
la complejidad algorítmica y las demostraciones de correción para algoritmos.
Esta última combinaría programación con matemáticas
y podría no ser impartida en el primer curso.

\footnotetext{
    A pesar de que la asignatura está asignada al área de
    Ingeniería de la Información y las Comunicaciones,
    los contenidos (lógica formal) son matemáticos.
}

\subsection{Herramientas}

% Mención a
% Git <- Posible anexo con ventajas para los profesores
% LaTex
% Debugging (gdb) <- ¿Mencionar los seminarios de Laguna y charlas de DAFI?

El último área que vamos a comentar en nuestro enfoque de primer curso
es el de las Tecnologías de la Información.
Una queja común entre los estudiantes del grado es que
no aprenden herramientas que son muy útiles durante la carrera.
En concreto, los estudiantes se quejan de no aprender
Git, \LaTeX{} y técnicas de debugging en primer curso.
Actualmente, estas técnicas apenas cobran importancia y, cuando se abordan,
es demasiado tarde\footnote{En el grado actual, el uso de Git se enseña
en la asignatura de tercero \subject{Tecnologías de Desarrollo Software};
el uso de \LaTeX{} se enseña en la asignatura optativa de tercero
\subject{Tecnologías Específicas de la Ingeniería Informática} cuya alternativa
es la asignatura de \subject{Prácticas Externas} que es el primer acercamiento
de los estudiantes al mundo laboral.}

Git es un sistema de control de versiones (SCV).
Saber utilizar un SCV es imprescindible en el mundo laboral;
pero además,
es una herramienta que facilitaría a los estudiantes la coordinación para
el desarrollo de las prácticas,
debido a que la mayoría se que se realizan en grupos de, al menos, dos personas.
Por eso mismo, los estudiantes creen que enseñarlo en primero
sería útil de cara al resto del grado.
Aunque hay otros SCV como SVN,
Git es, con gran diferencia, el SCV más usado y el estándar de facto.

\LaTeX, basado en \TeX, es un lenguaje para
la redacción documentos profesionales\footnote{
    Este documento, por ejemplo, ha sido creado con \LaTeX.
}.
Podría decirse que es prácticamente la única alternativa
libre y gratuita para la maquetación de textos.
Además, \LaTeX{} es ampliamente utilizado en el mundo académico y prueba de
ello es el hecho de que una buena parte de la documentación actual de las
 asignaturas del grado está desarrollada en él.
Los estudiantes usarían \LaTeX{} para redactar las memorias de sus trabajos.

Por último, a los estudiantes les gustaría aprender a depurar sus programas.
Es decir, aprender las tecnologías necesarias para hacerlo.
Actualmente, esta es una competencia que se ve de una forma muy breve y fugaz
en ciertas asignaturas donde apenas se le dedica tiempo y que, a su vez, el
dominarla supondría un gran beneficio para los estudiantes.
Desde el área estudiantil creemos que los estudiantes podrían empezar a
hacerlo desde primero.
Una buena forma sería
aprendiendo a utilizar herramientas como gdb y Valgrind\footnote{Actualmente,
 el uso de esta herramienta se enseña en tercero, en la asignatura de 
 \subject{Ampliación de Sistemas Operativos} y resultaría muy beneficioso el
 aprender a utilizarla mucho antes.} para depurar sus programas en C.

% TODO 4 abril
%--------------- BEGIN SEGUNDO --------------
\section{Segundo curso}

Como ya hemos comentado, el primer y segundo curso de un grado deben proporcionar una base sólida
de conocimientos al estudiante. En primero se deben impartir al estudiante los conocimientos 
básicos y comunes para los distintos grados de la rama de ingeniería, y segundo se debe 
centrar en dar la base de conocimientos específica para la ingeniería que el estudiante vaya a cursar.

En nuestro caso esto queda patente en la planificación de estudios actual para segundo; en la cual 
ya se tocan una gran variedad de ramas de la informática: algoritmia, estructuras de datos, 
programación orientada a objetos, programación concurrente, procesadores de lenguajes, 
redes, sistemas operativos, bases de datos y estructura de computadores.
Desde el área estudiantil estamos totalmente de acuerdo con las ramas escogidas, sin embargo,
creemos que en algunos casos se podría llegar a sacar más partido de algunas de las 
asignaturas.

Varios de los problemas y faltas que hemos podido detectar por nuestra experiencia 
en las asignaturas del segundo curso, ya han sido puestos de manifiesto en las propias propuestas 
de las áreas, por ejemplo: la necesidad de que el alumno sepa integrar una base de datos 
con el software de aplicación, como se propone desde el área de Lenguajes y Sistemas; u
otras asignaturas, como \subject{Programación Concurrente y Distribuida} creemos que podrían
beneficiarse de reducir su temario, y centrarse el temario de concurrencia pudiendo añadir a 
sus contenidos una iniciación al uso de la tecnología \textbf{OpenMP}.

\subsection{Competencias en programación}
Siguiendo la idea que hemos descrito en primero, creemos que el objetivo principal de
segundo debe ser dotar al estudiante de la actitud crítica y la capacidad de implementar, mediante
un código más limpio y matenible, distintos sistemas optimizando a su vez distintos 
recursos de la máquina: tiempo de ejecución, memoria, llamadas al sistema...

Otros aspectos que nos parecen relevantes para aquel estudiante que haya finalizado 
el segundo curso del grado son, junto con los expuestos en  el apartado superior: 
el manejo de excepciones para la creación de código robusto, el capacidad de usar la concurrencia 
en todos aquellos lenguajes de programación que conozca el estudiante, el dominio de más 
de un paradigma de programación, entre otros.

\subsection{Herramientas}
Debido a que segundo de grado es la iniciación a ramas de la informática tan diversas, 
no es de extrañar que el estudiantado deba instalar en su ordenador distintos programas y librerías;
a veces incompatibles con algunas tecnologías ya existentes dentro del propio ordenador del estudiante y 
produciendo de esta manera muchos problemas y pérdidas de tiempo por parte del estudiantado.

Por esta razón, desde el área estudiantil creemos que el uso de ciertas tecnologías, con creciente auge 
en el mundo laboral, como contenedores (Docker) y las máquinas virtuales deben ser fundamentales 
a partir de este curso y en adelante.
Además de esto, se debe seguir haciendo hincapié en las tecnologías ya comentadas en primero.

Se puede argumentar que muchos de los problemas de configuración y pérdidas de tiempo se pueden sortear
haciendo uso de los escritorios virtuales (EVA). Sin embargo, esto implica que el alumno deba 
tener una buena conexión a internet durante el tiempo estimado que vaya a trabajar en el proyecto; 
lo cual es una desventaja a la hora de usar esta tecnología, ya que el trabajo del estudiante 
va a estar supeditado a factores externos que este no tiene porqué controlar. Por ello, entre 
otras razones, pensamos que estas alternativas de trabajo \textit{offline} son tan necesarias.

%--------------- END SEGUNDO --------------

% TODO 4 abril
\section{Tercer curso}
A diferencia de los primeros cursos, donde nos centramos en contenidos generales 
y en fomentar habilidades que pueden ser comunes a todas las ramas del conocimiento de 
la informática, en este tercer año, se deberían tratar las ramas concretas de la 
informática a un nivel avanzado, proporcionando al estudiante un amplio abanico de 
conocimiento específico y obteniendo cierta soltura profesional.

En este sentido y, como ya hemos comentado antes, al finalizar este curso todo estudiante, 
independientemente de las optativas que curse tanto en este curso como en los posteriores, 
debería tener la base necesaria para poder afrontar un proyecto de cierta envergadura que, 
además, se podría desarrollar entre varias asignaturas\footnote{Actualmente se lleva a cabo
algo similar en tercero con las asignaturas de \subject{Procesos de Desarrollo Software} y 
\subject{Gestión de Procesos de Desarrollo Software} en las cuales se comparte un caso de 
estudio pero no se llega a abordar como tal un proyecto grande, solo su planificación.}; 
o realizar unas prácticas externas con soltura en la empresa u organismo en el que tengan 
lugar, con conocimientos de Computación, funcionamiento de redes y programación segura de 
aplicaciones.

Siguiendo la planificación planteada, además, en este curso el estudiante escogería la mención 
o algunas asignaturas optativas que lo focalizarán en las áreas que más le interesen.

\subsection{Competencias avanzadas}
Se debería de poder aprovechar y dar continuidad, de algún modo, a asignaturas que se 
imparten en los primeros cursos, en este caso, asignaturas como \subject{Estadística} o 
\subject{Bases de Datos}.

Por otro lado, a esta altura se debería tener el conocimiento necesario sobre computadores 
para poder tener unas mínimas nociones de optimización de rendimiento según el computador 
en el que trabajamos. En este sentido, si se diera el traspaso de algunos conocimientos de 
\subject{Arquitectura y Organización de Computadores} a \subject{Ampliación de Estructura 
de Computadores} se podría profundizar en algunos de los aspectos relacionados con estas
asignaturas. Además, a estas alturas el estudiante ya debería tener soltura y nociones 
avanzadas con la terminal.

En lo que respecta a la Ingeniería del Software, los conocimientos que se imparten en las 
asignaturas como \subject{Procesos de Desarrollo Software} y \subject{Gestión de Proyectos 
de Desarrollo de Software} se enfocan desde un punto de vista demasiado teórico de modo que, 
en algunos casos, se termina convirtiendo en un trabajo de memorización que es fácilmente 
olvidable al poco tiempo dado que no se llega a aplicar de ninguna forma. Sería conveniente 
que estas competencias se aglutinaran y se les diera un enfoque más práctico como podría ser, 
por ejemplo, el desarrollo de un proyecto utilizando metodologías ágiles como puede ser Scrum, 
que a su vez están ampliamente extendidos en el sector.

En lo que respecta a rama de redes y telemática, aunque se destaca la buena gestión 
que se realiza en las asignaturas de redes, son ideales para plantear escenarios donde
se fomenta el uso de contenedores y revisar el contenido repetido que a veces se produce
entre algunas de estas asignaturas.

Finalmente, en lo que respecta a la Computación, pensamos que asignaturas como 
\subject{Sistemas inteligentes} se perciben como obsoletas. El pensamiento lógico que se ve 
en los primeros temas puede ser tratado en cursos previos, mientras que además de ver 
sistemas basados en reglas, se podría realizar una introducción al Deep Learning o Machine 
Learning, así como una introducción a las redes neuronales y un primer contacto con este 
campo. De modo que los estudiantes que no cursen asignaturas de esta rama, tengan una base previa, 
además de darle sentido al fomento matemático que se hace en los primeros cursos.


\subsection{Base para el mundo laboral}
Asimismo, en este curso se podrían impartir seminarios sobre las nuevas tendencias y 
tecnologías de la ingeniería informática, de cara a tener un conocimiento previo, aunque 
no detallado, preparando y motivando al estudiante hacia su entrada en el mercado laboral o 
posibles líneas de investigación a explorar. Por lo que asignaturas como 
\subject{Tecnologías Específicas de la Ingeniería Informática (TEII)} podrían canalizar 
esta demanda de conocimiento, puesto que gran parte de su contenido ya habría sido adquirido 
por el alumnado en cursos previos.

Por último, hay que focalizar aún más el sentido práctico del grado, por lo que no se 
entiende que una carrera con tanta aplicación carezca de obligatoriedad las prácticas en 
empresa, pudiendo darse también poder hacerlas dentro de un grupo de investigación, 
adquiriendo el estudiante contacto con el mundo empresarial o de investigación. La mayor parte 
de los estudiantes que realizan estas prácticas coinciden en que fue muy motivador implicarse en 
proyectos reales y ver el funcionamiento y métodos de trabajo que se estudian en la carrera.
\section{Cuarto curso}

