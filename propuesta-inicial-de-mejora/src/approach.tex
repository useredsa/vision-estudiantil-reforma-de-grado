\chapter{Enfoque de la reforma de grado}\label{chap:approach}

\section{Introducción}

% Párrafo de introducción.
% Se pueden dar muchos enfoques a una reforma de grado.
% Este capítulo muestra el de la comisión de estudiantes.

Una reforma de grado es un proceso burocrático que implica a
profesores de cada una de las áreas y a alumnos de la titulación.
Con un equipo tan grande,
es importante comunicar los objetivos individuales para
tener una meta común clara.
En este capítulo damos a conocer el enfoque del área estudiantil de la comisión.

% Descripción del capítulo.

En primer lugar hablaremos de nuestros objetivos.
Después dejaremos claras nuestras opiniones respecto a
las tecnologías que deben enseñarse en el grado y
cómo debe manejarse la optatividad.
En el resto del capítulo
haremos una descripción de
los contenidos que queremos que tenga cada curso del grado.

% Objetivos del capítulo.
% Comunicar nuestra opinión al resto de la comisión.
% Justificar nuestra propuesta de grado.

El primer objetivo del capítulo es
dar a conocer al resto de la comisión nuestra opinión
para poder después integrarla con la suya.
El segundo es mostrar el enfoque que
hay detrás de las reformas que se describen en el \cref{chap:proposal}.

% TODO 4 abril
\section{Objetivos}

% TODO 4 abril
\section{Tecnologías}

Una de las cuestiones que queremos abordar es
la elección de tecnologías en el grado.
A lo largo del texto utilizaremos la palabra \emph{tecnología} para referirnos a
las herramientas que utilizan los ingenieros informáticos en su trabajo.
Ejemplos de tecnologías son
los lenguajes de programación;
como C, Java o Python;
librerías y frameworks de código;
que normalmente van asociadas a un lenguaje de programación;
productos como simuladores, depuradores y sistemas gestores de bases de datos; y
otras herramientas como sistemas de control de versiones.
Una tecnología tiene una aplicación de cara a un problema.
Por tanto, podemos agruparlas según su propósito.

Para un ingeniero informático,
conocer las tecnologías es de vital importancia,
especialmente en el mundo laboral,
donde dominar unas u otras abre la puerta a diferentes empleos.
En su formación,
un ingeniero informático aprende a utilizar varias tecnologías,
y en las universidades recae la tarea de decidir cuáles serán.

El día a día muestra que
una vez se ha utilizado una tecnología que resuelve un problema,
es mucho más sencillo aprender a utilizar otra que resuelva el mismo problema.
Por tanto, es común que los profesores argumenten que
no es tan relevante qué tecnologías se enseñe en el grado,
que el énfasis debe estar en aportar una enseñanza atemporal,
que permita al alumno encontrar trabajo durante toda su vida laboral,
independientemente de que las tecnologías predominantes cambien en el futuro.
Al contrario que los profesores,
los alumnos suelen tener interés en aprender las tecnologías que
les aporten más oportunidades en el mundo laboral actual.

Desde el punto de vista lógico,
si elegir una tecnología u otra no es determinante
en las habilidades que se adquieren,
pero aprender las nuevas ofrece más oportunidades laborales,
es estrictamente mejor aprender las nuevas.
Sin embargo, cambiar la tecnología utilizada
supone actualizar todo el material,
mejorado cuatrimestre a cuatrimestre,
de una asignatura,
lo cual es una tarea costosa para la universidad.

Reconociendo que la actualización de las tecnologías es necesario,
la Comisión de Análisis del Grado en Ingeniería Informática
determinó que era necesario
reducir el nivel de especificación de las tecnologías en las guías docentes,
ya que impide que el grado pueda adaptarse a los cambios en el mundo laboral.
El Área Estudiantil de
la Comisión de Reforma del Grado en Ingeniería Informática
también defiende que
el grado debe centrarse en enseñar esas habilidades básicas,
que no dependen de la tencología que se use,
pero que es esencial que se mantenga actualizado al mundo laboral actual.
Y que por tanto, es necesario actualizar algunas de las asignaturas del grado.
Además, que es importante elegir, de cara a los próximos años,
las tecnologías que se enseñarán y unificarlas para que
sean las mismas en diferentes asignaturas.
Y que debe considerarse en la comisión la opción de hacer el grado más flexible,
permitiendo a los alumnos elegir qué tecnologías utilizar
(en especial en los últimos cursos).

% Mención a la atemporalidad
% Es decir, cómo se va a manejar en las guías docentes
% para que las tecnologías puedan ir adaptándose.

\section{Optatividad}

Uno de los principales problemas que tiene que abordar
la Comisión de Reforma
es cómo reducir la optatividad del grado. %TODO (\cref{sec:optativity-problem}).
Además, la comisión también valora mover parte de la optatividad a tercer curso.

Desde el Área Estudiantil creemos que, efectivamente,
es razonable mover algunas de las asignaturas de las menciones a tercero para
incrementar el período de especialización
(y dar mayor experiencia en el área al salir del grado)
y para poder hacer espacio para
incorporar algunas asignaturas obligatorias en cuarto,
como solicitan algunos docentes.

En cuanto a la reducción en la oferta de optatividad,
creemos que es muy difícil mantener las $5$ menciones actuales
con solo $25$ asignaturas optativas (en lugar de $40$).
Nos gustaría apoyar la idea de crear una oferta abierta de asignaturas
y dejar a los alumnos elaborar su propio currículum.
Como punto a favor, creemos que
sería más fácil ofrecer $25$ asignaturas a elegir
que intentar que las menciones compartan asignaturas.
Como punto en contra, habría que decidir
cómo se cumpliría con la oferta de cupos,
lo cual se podría organizar para que
se asignase asignaturas a los estudiantes según su media;
y cómo se distribuiría el horario,
pues los estudiantes no podrían asistir a dos asignaturas que se solapasen.
En conjunto, creemos que es una opción que hay que estudiar.
Sin embargo, otras alternativas pueden también pueden funcionar bien.

\section{Primer curso}

El primer y segundo curso son normalmente la base de un grado,
porque son los cursos donde se instruye al estudiante en ramas que desconoce y
en las que se profundizará más adelante.

En el área estudiantil creemos que primero es un curso poco aprovechado
porque no se trabaja suficiente materia.
En el títulno actual, una persona que finalice primero
no tiene suficiente experiencia programando.
Por ende, hace más difíciles asignaturas de segundo curso
que requieren esa experiencia, como
\subject{Algoritmos y Estructuras de Datos},
\subject{Programación Orientada a Objetos} o
\subject{Programación Concurrente y Distribuida}.

Creemos que el primer curso del grado debe centrarse en
enseñar a los alumnos una buena base de
programación,
herramientas informáticas y
matemáticas.
Creemos que invertir más créditos de primero en estas categorías
dotará a los alumnos de mejor base y
mejorará su rendimiento durante el resto del grado. 
Además, estas categorías son también importantes en otras carreras,
como el resto de Ingenierías, Matemáticas o Física.
Por tanto, también dotan al estudiante de una buena base
en el caso de que decida cambiar de carrera.

\subsection{Competencias en programación}

Creemos que el principal objetivo del primer curso debe ser
dar al alumno una gran experiencia programando.
Para cumplirlo, la mayor parte del tiempo debe dedicarse,
precisamente, a programar.
Desde el área estudiantil diríamos que
en torno un \SI{50}{\percent} del tiempo.
El resultado esperado es que la experiencia que acumulen los alumnos
les permita razonar en las asignaturas que verán más adelante,
como \subject{Programación Orientada a Objetos} o
\subject{Algoritmos y Estructuras de Datos}\footnotemark.

\footnotetext{O aquellas que enseñen esos contenidos en el nuevo plan.}

El nivel adquirido tras
haber superado satisfactoriamente las asignaturas de primer curso
debería ser suficiente para implementar ideas complejas en
un lenguaje de programación iterativo
interactuando con la entrada y la salida estándar.
Esto podría ser un resultado del aprendizaje que habría que definir,
aunque en este punto de la carrera,
no se exigiría que las soluciones que encontrasen fuesen las mejores en
complejidad, estructura y prácticas de programación
(como limpieza de código).
El \cref{tab:first-course-programming-examples}
muestra una serie de problemas que deberíamos conseguir que
un alumno de primero pudiese implementar,
junto con una estimación del tiempo necesario para hacerlo.
Dos tercios del cuadro podrían abordarse en
una asignatura de programación durante el primer cuatrimestre.

\begin{table}[h]
    \centering
    \begin{tabular}{ll}
        \textbf{Tiempo} & \textbf{Problema} \\
        \hline
        \multirow{3}{*}{Hasta \SI{30}{min}.}
        & Escribir un número en otra base. \\
        & Hallar la transpuesta de una matriz. \\
        & Sumar fracciones. \\
        \hline
        \multirow{4}{*}{De \SIrange{30}{50}{min}.}
        & Factorizar un número. \\ 
        & Ordenar una secuencia de números. \\
        & Encontrar las ocurrencias de una cadena en un crucigrama. \\
        & Comprobar si un cuadrado $9\times 9$ es solución de un sudoku. \\
        \hline
        \multirow{3}{*}{De \SIrange{1}{3}{h}.}
        & Determinar si un elemento se encuentra en un array ordenado. \\
        & Hallar la unión de dos arrays ordenados. \\
        & Evaluar s-expresiones de la forma \lstinline!* 4 + 1 2!.
    \end{tabular}
    \caption{
        Competencias mínimas en programación tras superar el primer curso.
        Para cada problema, se incluye el tiempo esperado para resolverlo.
    }
    \label{tab:first-course-programming-examples}
\end{table}

\subsection{Matemáticas}

El otro área en el que hay que enfatizar en primero son las matemáticas.
La arquitectura de computadores, la ciberseguridad y la computación
son áreas de la ingeniería informática donde
las matemáticas son imprescindibles.
El resto de áreas, como la ingeniería del software,
también necesitan mentes resolutivas y con buena capacidad de abstracción.
De hecho, uno de los problemas que
identificó la Comisión de Análisis del Grado
fue que los alumnos no tenían una buena base matemática.

De igual manera,
las matemáticas también son la base del resto de ingenierías.
Concentrar mayor parte de las asignaturas de matemáticas en el primer curso
también facilitaría a los alumnos cambiar de carrera
aprovechando parte de los créditos aprobados.

Las materias que se imparten en primero del grado actualmente son
\subject{Álgebra y Matemática Discreta},
\subject{Fundamentos Lógicos de la Informática}\footnotemark,
\subject{Cálculo} y
\subject{Estadística}.
Por un lado, nuestra opinión es que
la cantidad de créditos de matemáticas en primero es correcta.
Sin embargo, creemos que la asignatura de lógica
podría ser sustituida por otra asignatura de matemáticas.
Además, la asignatura \subject{Álgebra y Matemática Discreta}
podría dividirse en dos asignaturas,
una de Álgebra Lineal y otra que tratase
la complejidad algorítmica y las demostraciones de correción para algoritmos.
Esta última combinaría programación con matemáticas
y podría no ser impartirse en el primer curso.

\footnotetext{
    A pesar de que la asignatura está asignada al área de
    Ingeniería de la Información y las Comunicaciones,
    los contenidos (lógica formal) son matemáticas.
}

\subsection{Herramientas}

% Mención a
% Git <- Posible anexo con ventajas para los profesores
% LaTex
% Debugging (gdb)

El último área que vamos a comentar en nuestro enfoque de primer curso
es el de las Tecnologías de la Información.
Una queja común entre los alumnos del grado es que
no aprenden herramientas que son muy útiles durante la carrera.
En concreto, los alumnos se quejan de no aprender
Git, \LaTeX{} y técnicas de debugging en primer curso.

Git es un sistema de control de versiones (SCV).
Saber utilizar un SCV es imprescindible en el mundo laboral;
pero además,
es una herramienta que facilita a los alumnos colaborar para las prácticas,
que son en parejas.
Por eso mismo, los alumnos creen que enseñarlo en primero
es útil de cara al resto del grado.
Aunque hay otros SCV como SVN,
Git es, con gran diferencia, el SCV más usado y el estándar de facto.

\LaTeX, basado en \TeX, es un lenguaje para
la redacción documentos profesionales\footnote{
    Este documento, por ejemplo, ha sido creado con \LaTeX.
}.
Podría decirse que es prácticamente la única alternativa
libre y gratuita para la maquetación de textos.
Además, \LaTeX{} es ampliamente utilizado en el mundo académico.
Los estudiantes usarían \LaTeX{} para redactar las memorias de sus trabajos.

Por último, a los alumnos les gustaría aprender a depurar sus programas.
Es decir, aprender las tecnologías necesarias para hacerlo.
Desde el área estudiantil creemos que los alumnos podrían empezar a
hacerlo desde primero.
Una buena forma sería
aprendiendo a utilizar gdb para depurar sus programas en C.

% TODO 4 abril
\section{Segundo curso}

% TODO 4 abril
\section{Tercer curso}

\section{Cuarto curso}

