\chapter{Enfoque de la reforma de grado}

\section{Introducción}

% Párrafo de introducción.
% Se pueden dar muchos enfoques a una reforma de grado.
% Este capítulo muestra el de la comisión de estudiantes.

Una reforma de grado es un proceso burocrático que implica a
profesores de cada una de las áreas y a alumnos del grado.
Con un equipo tan grande,
es importante comunicar los objetivos individuales para
tener un objetivo común claro.
En este capítulo damos a conocer el enfoque del área estudiantil de la comisión.

% Descripción del capítulo.

En primer lugar hablaremos de nuestros objetivos.
Después dejaremos claras nuestras opiniones respecto a
las tecnologías que deben enseñarse en el grado y
cómo debe manejarse la optatividad.
En el resto del capítulo
haremos una descripción de los objetivos que debe tener cada curso del grado.

% Objetivos del capítulo.
% Comunicar nuestra opinión al resto de la comisión.
% Justificar nuestra propuesta de grado.

El primer objetivo del capítulo es
dar a conocer al resto de la comisión nuestra opinión
para poder después integrarla con la suya.
El segundo es mostrar el enfoque que
hay detrás de las reformas que se describen en el \cref{chap:proposal}.

% TODO 4 abril
\section{Objetivos}

% TODO 4 abril
\section{Tecnologías}

% Mención a la atemporalidad
% Es decir, cómo se va a manejar en las guías docentes
% para que las tecnologías puedan ir adaptándose.

%TODO 4 abril
\section{Optatividad}

% TODO 4 abril
\section{Primer curso}

% Mención a
% Git <- Posible anexo con ventajas para los profesores
% LaTex
% Debugging (gdb)

% TODO 4 abril
\section{Segundo curso}

% TODO 4 abril
\section{Tercer curso}

\section{Cuarto curso}

