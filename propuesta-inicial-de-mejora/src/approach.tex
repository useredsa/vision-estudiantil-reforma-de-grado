\chapter{Enfoque de la reforma de grado}

\section{Introducción}

% Párrafo de introducción.
% Se pueden dar muchos enfoques a una reforma de grado.
% Este capítulo muestra el de la comisión de estudiantes.

Una reforma de grado es un proceso burocrático que implica a
profesores de cada una de las áreas y a alumnos de la titulación.
Con un equipo tan grande,
es importante comunicar los objetivos individuales para
tener una meta común clara.
En este capítulo damos a conocer el enfoque del área estudiantil de la comisión.

% Descripción del capítulo.

En primer lugar hablaremos de nuestros objetivos.
Después dejaremos claras nuestras opiniones respecto a
las tecnologías que deben enseñarse en el grado y
cómo debe manejarse la optatividad.
En el resto del capítulo
haremos una descripción de
los contenidos que queremos que tenga cada curso del grado.

% Objetivos del capítulo.
% Comunicar nuestra opinión al resto de la comisión.
% Justificar nuestra propuesta de grado.

El primer objetivo del capítulo es
dar a conocer al resto de la comisión nuestra opinión
para poder después integrarla con la suya.
El segundo es mostrar el enfoque que
hay detrás de las reformas que se describen en el \cref{chap:proposal}.

% TODO 4 abril
\section{Objetivos}

% TODO 4 abril
\section{Tecnologías}

Una de las cuestiones que queremos abordar es
la elección de tecnologías en el grado.
A lo largo del texto utilizaremos la palabra \emph{tecnología} para referirnos a
las herramientas que utilizan los ingenieros informáticos en su trabajo.
Ejemplos de tecnologías son
los lenguajes de programación;
como C, Java o Python;
librerías y frameworks de código;
que normalmente van asociadas a un lenguaje de programación;
productos como simuladores, depuradores y sistemas gestores de bases de datos; y
otras herramientas como sistemas de control de versiones.
Una tecnología tiene una aplicación de cara a un problema.
Por tanto, podemos agruparlas según su propósito.

Para un ingeniero informático,
conocer las tecnologías es de vital importancia,
especialmente en el mundo laboral,
donde dominar unas u otras abre la puerta a diferentes empleos.
En su formación,
un ingeniero informático aprende a utilizar varias tecnologías,
y en las universidades recae la tarea de decidir cuáles serán.

El día a día muestra que
una vez se ha utilizado una tecnología que resuelve un problema,
es mucho más sencillo aprender a utilizar otra que resuelva el mismo problema.
Por tanto, es común que los profesores argumenten que
no es tan relevante qué tecnologías se enseñe en el grado,
que el énfasis debe estar en aportar una enseñanza atemporal,
que permita al alumno encontrar trabajo durante toda su vida laboral,
independientemente de que las tecnologías predominantes cambien en el futuro.
Al contrario que los profesores,
los alumnos suelen tener interés en aprender las tecnologías que
les aporten más oportunidades en el mundo laboral actual.

Desde el punto de vista lógico,
si elegir una tecnología u otra no es determinante
en las habilidades que se adquieren,
pero aprender las nuevas ofrece más oportunidades laborales,
es estrictamente mejor aprender las nuevas.
Sin embargo, cambiar la tecnología utilizada
supone actualizar todo el material,
mejorado cuatrimestre a cuatrimestre,
de una asignatura,
lo cual es una tarea costosa para la universidad.

Reconociendo que la actualización de las tecnologías es necesario,
la Comisión de Análisis del Grado en Ingeniería Informática
determinó que era necesario
reducir el nivel de especificación de las tecnologías en las guías docentes,
ya que impide que el grado pueda adaptarse a los cambios en el mundo laboral.
El Área Estudiantil de
la Comisión de Reforma del Grado en Ingeniería Informática
también defiende que
el grado debe centrarse en enseñar esas habilidades básicas,
que no dependen de la tencología que se use,
pero que es esencial que se mantenga actualizado al mundo laboral actual.
Y que por tanto, es necesario actualizar algunas de las asignaturas del grado.
Además, que es importante elegir, de cara a los próximos años,
las tecnologías que se enseñarán y unificarlas para que
sean las mismas en diferentes asignaturas.
Y que debe considerarse en la comisión la opción de hacer el grado más flexible,
permitiendo a los alumnos elegir qué tecnologías utilizar
(en especial en los últimos cursos).

% Mención a la atemporalidad
% Es decir, cómo se va a manejar en las guías docentes
% para que las tecnologías puedan ir adaptándose.

%TODO 4 abril
\section{Optatividad}

% TODO 4 abril
\section{Primer curso}

% Mención a
% Git <- Posible anexo con ventajas para los profesores
% LaTex
% Debugging (gdb)

% TODO 4 abril
\section{Segundo curso}

% TODO 4 abril
\section{Tercer curso}

\section{Cuarto curso}

