\begin{SubjectProposal}
    \begin{SubjectTabular}
        \Semester{1} \Ects{9}
        \SubjectName{Programación I}
        \SubjectType{Formación Básica}
        \SubjectArea{Programación}
        \SubjectDescription{
            \begin{itemize}[leftmargin=*]
                \item Conceptos de algoritmo y programa
                \item Entrada y salida estándar
                \item Programación imperativa
                \begin{itemize}
                    \item Uso de la entrada y la salida estándar
                    \item Condicionales
                    \item Bucles
                    \item Arrays
                    \item Arrays multidimensionales
                    \item Funciones
                    \item Recursión
                \end{itemize}
                \item Desarrollo de algoritmos fundamentales y
                pruebas de correción
                \begin{itemize}
                    \item Sumas acumuladas
                    \item Búsqueda binaria
                    \item Ordenación por el método de la burbuja
                    \item Recorridos en profundidad recursivos en tablas
                    \item Tratamiento de secuencias
                    \item Algoritmos ad-hoc sencillos
                \end{itemize}
                \item Complejidad algorítmica
            \end{itemize}
        }
    \end{SubjectTabular}

    \begin{SubjectTabular}
        \Semester{2} \Ects{9}
        \SubjectName{Programación II}
        \SubjectType{Formación Básica}
        \SubjectArea{Programación}
        \SubjectDescription{
            \begin{itemize}[leftmargin=*]
                \item Aspectos avanzados del lenguaje en uso,
                modularización
                \item Principios de encapsulación y ocultación de la información
                \item Resolución de problemas algorítmicos usando
                tipos abstractos de datos:
                arrays de tamaño variable, colas, pilas y conjuntos ordenados
                \item Implementación de los tipos abstractos de datos:
                arrays de tamaño variable, colas, pilas, árboles
                \item Gestión de la memoria en lenguajes de sistemas
                \item Tipos abstractos de datos para la gestión de la memoria
                en proyectos modernos: \textit{smart pointers}
            \end{itemize}
        }
    \end{SubjectTabular}

    \caption{Las dos asignaturas de programación de primer curso}
    \label{spr:programacion-i-ii}
\end{SubjectProposal}

\begin{SubjectProposal}
    \begin{SubjectTabular}
        \SubjectName{Manejo de Shell y Herramientas Básicas de Programación}
        \Semester{[1-2]} \Ects{6}
        \SubjectType{Formación Básica}
        \SubjectArea{Fundamentos de la Informática}
        \SubjectDescription{
            \begin{itemize}[leftmargin=*]
                \item Introducción a Linux
                \item Manejo del intérprete de órdenes
                \item Herramientas de desarrollo:
                editor, compilador, enlazador, depurador,
                sistemas de control de versiones y
                herramientas para la compilación de proyectos.
                \item Jerarquía de traducción y generación de código.
                \item Automatización de tareas mediante programación de scripts.
                \item Herramientas de documentación
            \end{itemize} \\
            \textbf{Posibles seminarios}           &
            \begin{itemize}[leftmargin=*]
                \item Instalación de Linux y uso básico de la terminal
                \item Introducción a LaTeX
            \end{itemize}
        }
    \end{SubjectTabular}
    \caption{Asignatura básica de introducción a las herramientas informáticas.}
    \label{spr:manejo-shell-herramientas}
\end{SubjectProposal}

\begin{SubjectProposal}
    \begin{SubjectTabular}
        \SubjectName{Fundamentos de Computadores}
        \Semester{2} \Ects{6}
        \SubjectType{Formación Básica}
        \SubjectArea{Fundamentos de}
        \SubjectDescription{
            \begin{itemize}[leftmargin=*]
                \item Introducción:
                evolución y desarrollo histórico de los computadores,
                esquema de funcionamiento de un ordenador
                (arquitectura von Neumann)
                y arquitectura básica de un Computador.
                \item Representación de la información:
                datos de tipo entero, real y carácter,
                operaciones con los datos básicos,
                nociones básicas sobre representación de datos complejos.
                \item Sistemas digitales:
                Álgebra de Boole, circuitos combinacionales,
                circuitos secuenciales y circuitos aritmético-lógicos.
            \end{itemize} \\
            \textbf{Posibles seminarios} & Seminario de introducción a las redes
        }
    \end{SubjectTabular}
    \caption{Asignatura de fundamentos de la informática}
    \label{spr:fundamentos-de-computadores}
\end{SubjectProposal}

\begin{SubjectProposal}
    \begin{SubjectTabular}
        \SubjectName{Álgebra Lineal}
        \Semester{1} \Ects{6}
        \SubjectType{Formación Básica}
        \SubjectArea{Fundamentos matemáticos}
        \SubjectDescription{
            \begin{itemize}[leftmargin=*]
                \item Matrices y sistemas de ecuaciones lineales
                \item Espacios vectoriales y aplicaciones lineales
                \item Sistemas de ecuaciones
                \item Diagonalización
                \item Algoritmos básicos de álgebra lineal
                \begin{itemize}
                    \item Resolución de sistemas de ecuaciones módulo $2$
                    \item Exponenciación rápida de matrices
                \end{itemize}
            \end{itemize}
        }
    \end{SubjectTabular}

    \begin{SubjectTabular}
        \SubjectName{Cálculo}
        \Semester{1} \Ects{6}
        \SubjectType{Formación Básica}
        \SubjectArea{Fundamentos matemáticos}
        \SubjectDescription{
            \begin{itemize}[leftmargin=*]
                \item Sucesiones y definición de límite de una sucesión
                \item Definición de límite de una función. Continuidad
                \item Derivabilidad e Integración
                \item Teorema de Bolzano y completitud de R
                \item Fórmula de Taylor
            \end{itemize}
        }
    \end{SubjectTabular}

    \begin{SubjectTabular}
        \SubjectName{Estadística}
        \Semester{2} \Ects{6}
        \SubjectType{Formación Básica}
        \SubjectArea{Fundamentos matemáticos}
        \SubjectDescription{
            \begin{itemize}[leftmargin=*]
                \item Introducción a la combinatoria
                \item Espacios de probabilidad discretos y
                variables aleatorias discretas.
                \item Esperanza matemática
                \item Espacios de probabilidad continuos y
                variables aleatorias continuas
                \item Estadística descriptiva
                \item Introducción a la inferencia estadística
            \end{itemize} \\
            \textbf{Posibles seminarios}           &
            Análisis de complejidad en un treap y/o
            algoritmos sobre cadenas con hashes
        }
    \end{SubjectTabular}

    \caption{Las tres asignaturas de fundamentos matemáticos de primer curso}
    \label{spr:first-course-mathematics}
\end{SubjectProposal}

\begin{SubjectProposal}
    \begin{SubjectTabular}
        \SubjectName{Gestión de Organizaciones y Habilidades Profesionales}
        \Semester{1} \Ects{6}
        \SubjectType{Formación Básica}
    \end{SubjectTabular}
    \caption{...} %TODO
    \label{spr:gohp}
\end{SubjectProposal}

\begin{SubjectProposal}
    \begin{SubjectTabular}
        \SubjectName{Fundamentos en electricidad y electrónica}
        \Semester{2} \Ects{6}
        \SubjectType{Formación Básica}
        \SubjectArea{Fundamentos Físicos}
        \SubjectDescription{
            \begin{itemize}[leftmargin=*]
                \item Fenómenos eléctricos
                \item Circuitos de corriente continua
                \item Fenómenos magnéticos y electromagnéticos
                \item Circuitos RC y RL
                \item Corriente alterna
                \item Diodos y transistores
                \item Amplificadores operacionales
            \end{itemize}
        }
    \end{SubjectTabular}
    \caption{Asignatura de física}
    \label{spr:fisica}
\end{SubjectProposal}

