% Code Listings Configuration File

\usepackage{listings}
\renewcommand{\lstlistingname}{Código}
\crefname{lstlisting}{código}{códigos}
\Crefname{lstlisting}{Código}{Códigos}


\definecolor{background}{rgb}{0.99,0.99,0.99}
\definecolor{mygreen}{rgb}{0.1,0.6,0.1}
\definecolor{mygray}{rgb}{0.5,0.5,0.5}
\definecolor{mymauve}{rgb}{0.58,0,0.82}
\definecolor{codeback}{rgb}{0.85,0,0.85}

\renewcommand{\lstlistlistingname}{Índice de códigos}
\lstset{
  title=Fichero \lstname,          % show the filename of files included with \lstinputlisting; also try caption instead of title
  caption=Fichero \texttt{\lstname},
  captionpos=t,                    % sets the caption-position (b/t)
  frame=TBLR,      	               % adds a frame around the code
  rulecolor=\color{black},         % if not set, the frame-color may be changed on line-breaks within not-black text (e.g. comments (green here))
  numbers=left,                    % where to put the line-numbers; possible values are (none, left, right)
  numbersep=10pt,                  % how far the line-numbers are from the code
  % stepnumber=1,                    % the step between two line-numbers. If it's 1, each line will be numbered
  backgroundcolor=\color{background},
  numberstyle=\small\color{mygray},% the style that is used for the line-numbers
  basicstyle=\scriptsize\selectfont, % the size/type of the fonts that are used for the code
  keywordstyle=\color{blue},       % keyword style
  commentstyle=\color{mygreen},    % comment style
  stringstyle=\color{mymauve},     % string literal style
  breakatwhitespace=false,         % sets if automatic breaks should only happen at whitespace
  breaklines=true,                 % sets automatic line breaking
  keepspaces=true,                 % keeps spaces in text, useful for keeping indentation of code (possibly needs columns=flexible)
  showstringspaces=false,          % underline spaces within strings only
  showspaces=false,                % show spaces everywhere adding particular underscores; it overrides 'showstringspaces'
  showtabs=false,                  % show tabs within strings adding particular underscores
  tabsize=2,	                   % sets default tabsize to 2 spaces
  language=C++,                    % the language of the code
  morekeywords={},            % if you want to add more keywords to the set
  deletekeywords={},            % if you want to delete keywords from the given language
  % escapeinside={----}{----},          % if you want to add LaTeX within your code
  extendedchars=true,              % lets you use non-ASCII characters; for 8-bits encodings only, does not work with UTF-8
  literate=
  {á}{{\'a}}1 {é}{{\'e}}1 {í}{{\'i}}1 {ó}{{\'o}}1 {ú}{{\'u}}1
  {Á}{{\'A}}1 {É}{{\'E}}1 {Í}{{\'I}}1 {Ó}{{\'O}}1 {Ú}{{\'U}}1
  {à}{{\`a}}1 {è}{{\`e}}1 {ì}{{\`i}}1 {ò}{{\`o}}1 {ù}{{\`u}}1
  {À}{{\`A}}1 {È}{{\'E}}1 {Ì}{{\`I}}1 {Ò}{{\`O}}1 {Ù}{{\`U}}1
  {ä}{{\"a}}1 {ë}{{\"e}}1 {ï}{{\"i}}1 {ö}{{\"o}}1 {ü}{{\"u}}1
  {Ä}{{\"A}}1 {Ë}{{\"E}}1 {Ï}{{\"I}}1 {Ö}{{\"O}}1 {Ü}{{\"U}}1
  {â}{{\^a}}1 {ê}{{\^e}}1 {î}{{\^i}}1 {ô}{{\^o}}1 {û}{{\^u}}1
  {Â}{{\^A}}1 {Ê}{{\^E}}1 {Î}{{\^I}}1 {Ô}{{\^O}}1 {Û}{{\^U}}1
  {œ}{{\oe}}1 {Œ}{{\OE}}1 {æ}{{\ae}}1 {Æ}{{\AE}}1 {ß}{{\ss}}1
  {ű}{{\H{u}}}1 {Ű}{{\H{U}}}1 {ő}{{\H{o}}}1 {Ő}{{\H{O}}}1
  {ç}{{\c c}}1 {Ç}{{\c C}}1 {ø}{{\o}}1 {å}{{\r a}}1 {Å}{{\r A}}1
  {€}{{\euro}}1 {£}{{\pounds}}1 {«}{{\guillemotleft}}1
  {»}{{\guillemotright}}1 {ñ}{{\~n}}1 {Ñ}{{\~N}}1 {¿}{{?`}}1
}
