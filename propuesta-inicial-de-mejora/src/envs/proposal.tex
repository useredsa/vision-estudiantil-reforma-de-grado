\usepackage{newfloat}
\usepackage{caption}

\DeclareFloatingEnvironment[
    fileext=frm,
    listname=Lista de asignaturas propuestas,
    name=Propuesta,
    placement={p},
    within=none,
]{SubjectProposal}

\crefname{SubjectProposal}{prop.}{props.}
\Crefname{SubjectProposal}{Prop.}{Props.}

\newenvironment{SubjectTabular}{\clearpage}{%
    \begin{tabular}{|p{3.7cm}p{10cm}|}
        \hline
        \textbf{Bloque de contenidos} & \givenSubjectName \\
        \textbf{Créditos}             & \givenEcts \\
        \textbf{Cuatrimestre}         & \givenSemester \\
        \textbf{Tipo}                 & \givenSubjectType \\
        \textbf{Materia asociada}     & \givenSubjectArea \\
        \textbf{Descripción}          & \givenSubjectDescription \\
        \hline
    \end{tabular}
    \vspace{0.1cm}
}

\newcommand{\givenSubjectName}{REQUIRED!! - (TODO)}
\newcommand{\givenEcts}{REQUIRED!! - (TODO)}
\newcommand{\givenSemester}{REQUIRED!! - (TODO)}
\newcommand{\givenSubjectType}{REQUIRED!! - (TODO)}
\newcommand{\givenSubjectArea}{REQUIRED!! - (TODO)}
\newcommand{\givenSubjectDescription}{REQUIRED!! - (TODO)}

\newcommand{\SubjectName}[1]{\renewcommand{\givenSubjectName}{#1}}
\newcommand{\Ects}[1]{\renewcommand{\givenEcts}{#1}}
\newcommand{\Semester}[1]{\renewcommand{\givenSemester}{#1}}
\newcommand{\SubjectType}[1]{\renewcommand{\givenSubjectType}{#1}}
\newcommand{\SubjectArea}[1]{\renewcommand{\givenSubjectArea}{#1}}
\newcommand{\SubjectDescription}[1]{\renewcommand{\givenSubjectDescription}{#1}}

