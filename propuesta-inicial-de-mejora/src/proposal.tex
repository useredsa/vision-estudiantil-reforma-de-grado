\chapter{Propuesta de Reforma de Grado}\label{chap:proposal}

% Ver propuestas de grado

% TODO 4 abril
\section{Planificación de grado}

% TODO 4 abril
\subsection{Planificación por cursos}

\section{Primer curso}

\begin{table}[h]
    \centering
    \begin{tabular}{crll}
        \textbf{\small Cuatr.} & \textbf{\small Créditos} & \textbf{Asignatura}
            & \multicolumn{1}{m{2cm}}{\textbf{Bloque de Contenidos}} \\
        \hline\hline
        1     & 9 & Programación I  & \Cref{spr:programacion-i-ii} \\
        1-2   & 6 & Manejo de Shell y Herramientas Básicas de Programación &
            \Cref{spr:manejo-shell-herramientas} \\
        1     & 6 & Álgebra Lineal  & \Cref{spr:first-course-mathematics} \\
        1     & 6 & Cálculo         & \Cref{spr:first-course-mathematics} \\
        1     & 6 & Gestión de Organizaciones y Habilidades Profesionales &
            \Cref{spr:gohp} \\
        2     & 9 & Programación II & \Cref{spr:programacion-i-ii} \\
        2     & 6 & Fundamentos de Computadores &
            \Cref{spr:fundamentos-de-computadores} \\
        2     & 6 & Estadística     & \Cref{spr:first-course-mathematics} \\
        2     & 6 & Fundamentos en electricidad y electrónica &
            \Cref{spr:fisica} \\
    \end{tabular}
    \caption{Asignaturas de primero}
    \label{tab:first-course-subjects}
\end{table}

\subsection{Descripción del curso}

El bloque de contenidos~(\cref{spr:programacion-i-ii})
más importante del primer curso son las asignaturas de programación.
Hemos querido aumentar de $12$ a $18$ el número de créditos
dedicados a la materia.
Aunque no en la cantidad,
coincidimos con el Departamento de Informática y Sistemas,
que pedía aumentar la carga a $15$ créditos.
La materia se dividiría en $2$ asignaturas de $9$ créditos.
Programación I y Programación II.

Programación I sería una introducción desde cero a la programación
para resolver problemas algorítmicos.
A lo largo de la misma los estudiantes aprenderían las nociones básicas
de un lenguaje de programación iterativo e
ideas algorítmicas sencillas que pueden aplicarse en muchos problemas,
como el cálculo de sumas acumuladas,
el algoritmo de la búsqueda binaria
o un método de ordenación sencillo.

En Programación II se continuaría enseñando el lenguaje,
trabajando por primera vez en proyectos divididos en varios ficheros.
En esta asignatura el estudiante aprendería
el concepto de tipo abstracto de dato,
e implementaría los suyos,
entendiendo qué significa programar de cara a la interfaz
y manejando manualmente la memoria dinámica.
Las nuevas estructuras permitirán al estudiante
abordar problemas algorítmicos más difíciles.

Como afirmábamos en el \cref{sec:approach-first-course},
el objetivo de las asignaturas debe ser
aportar la mayor cantidad de experiencia en este primer año.
Por eso, pensamos que las asignaturas
tienen que tener una carga lectiva de $\SI{5.5}{h}$ semanales,
de las cuales $\SI{2}{h}$ serían clases teóricas y el resto,
el grueso de la asignatura,
clases prácticas dedicadas a la resolución de problemas de programación
en los que se deba interactuar con la entrada y la salida estándar.

Desde el Área Estudiantil queremos sugerir el uso de un juez online\footnote{
    Un juez online es una herramienta con la que entrenarse programando
    que corrige automáticamente los programas enviados por los estudiantes.
} para la correción de los ejercicios.
Es algo que ya se usa en otras universidades españolas,
como la Universidad Politécnica de Cataluña o
la Universidad Complutense de Madrid y
que han solicitado bastantes estudiantes.
Un juez online aporta retroalimentación a los alumnos de manera instantánea
y les permite evaluar sus programas en casa sin la ayuda de un profesor.

Como complemento a las asignaturas de programación,
creemos que es necesario incluir
una asignatura enfocada en las herramientas de programación.
La asignatura Manejo de Shell y Herramientas Básicas de Programación~%
(\cref{spr:manejo-shell-herramientas})
es el resultado de fusionar dos asignaturas de $3$ créditos propuestas por
el área Arquitectura y Tecnología de Computadores.

En vez de tener dos asignaturas de $3$ créditos,
una en cada cuatrimestre,
esta asignatura anual tiene las siguientes ventajas
Al ser una asignatura anual,
podría dedicar varias semanas del primer cuatrimestre
en seminarios obligatorios,
e impartir más teoría en el segundo cuatrimestre.
Por ejemplo, en la segunda semana se podría impartir un seminario de
introducción a Linux y al uso de la terminal para compilar y ejecutar programas.
Nosotros vemos a la asignatura como una introducción a la carrera,
con un primer cuatrimestre enfocado en
enseñar las herramientas que utilizarán en la carrera
y un segundo cuatrimestre centrado en la programación de scripts.
Otro punto positivo de hacer esta asignatura anual es
reducir el número de exámenes en enero.
Los estudiantes tienen que adaptarse a la universidad en el primer cuatrimestre.
Pero además, con los cambios que proponemos,
incluímos Álgebra Lineal y Cálculo en el primer cuatrimestre.
Es decir, es un cuatrimestre difícil.
Eso justificaría hacer la primera parte de la asignatura más llevadera,
con multitud de seminarios y asistencia para las clases de Programación I,
la asignatura más importante del primer cuatrimestre.

El otro gran bloque de contenidos~(\cref{spr:first-course-mathematics})
de primero es el de fundamentos matemáticos.
El Departamento de Matemáticas Aplicada
ha propuesto un cambio en sus asignaturas de primer curso:
crear dos asignaturas de $6$ créditos nuevas,
Álgebra Lineal y Matemática Discreta,
en vez de la antigua Álgebra y Matemática Discreta,
también de $6$ créditos.
Además, ha querido desplazar Estadística a segundo curso,
dado que actualmente,
los estudiantes cursan Estadística en el segundo cuatrimestre,
al mismo tiempo que Cálculo.

Lo que al Área Estudiantil le gustaría son
ligeras modificaciones de esta propuesta.
Nosotros optamos por dar Álgebra Lineal y Cálculo
en el primer cuatrimestre de primero y
Estadística en el segundo.
Matemática Discreta se cursaría en segundo de carrera,
al mismo tiempo que asignaturas como Algoritmos y Estructuras de Datos,
con las que tiene mucha relación.
Con esto queda resuelto el problema que había con la asignatura de Estadística y
se mantiene constante el número de créditos de matemáticas en primero.
$18$ créditos, al igual la materia programación.
(Nótese que las asignaturas de matemáticas que
hemos seleccionado para el primer curso
son comunes, si no a todas, a la gran mayoría de las carreras STEM.)

El resto de asignaturas que incluímos en primero:
Fundamentos de Computadores;
Gestión de Organizaciones y Habilidades Profesionales; y
Fundamentos en electricidad y electrónica,
se encontraban ya en el plan antiguo y no han sufrido grandes modificaciones.

\begin{SubjectProposal}
    \begin{SubjectTabular}
        \Semester{1} \Ects{9}
        \SubjectName{Programación I}
        \SubjectType{Formación Básica}
        \SubjectArea{Programación}
        \SubjectDescription{
            \begin{itemize}[leftmargin=*]
                \item Conceptos de algoritmo y programa
                \item Entrada y salida estándar
                \item Programación imperativa
                \begin{itemize}
                    \item Uso de la entrada y la salida estándar
                    \item Condicionales
                    \item Bucles
                    \item Arrays
                    \item Arrays multidimensionales
                    \item Funciones
                    \item Recursión
                \end{itemize}
                \item Desarrollo de algoritmos fundamentales y
                pruebas de correción
                \begin{itemize}
                    \item Sumas acumuladas
                    \item Búsqueda binaria
                    \item Ordenación por el método de la burbuja
                    \item Recorridos en profundidad recursivos en tablas
                    \item Tratamiento de secuencias
                    \item Algoritmos ad-hoc sencillos
                \end{itemize}
                \item Complejidad algorítmica
            \end{itemize}
        }
    \end{SubjectTabular}

    \begin{SubjectTabular}
        \Semester{2} \Ects{9}
        \SubjectName{Programación II}
        \SubjectType{Formación Básica}
        \SubjectArea{Programación}
        \SubjectDescription{
            \begin{itemize}[leftmargin=*]
                \item Aspectos avanzados del lenguaje en uso,
                modularización
                \item Principios de encapsulación y ocultación de la información
                \item Resolución de problemas algorítmicos usando
                tipos abstractos de datos:
                arrays de tamaño variable, colas, pilas y conjuntos ordenados
                \item Implementación de los tipos abstractos de datos:
                arrays de tamaño variable, colas, pilas, árboles
                \item Gestión de la memoria en lenguajes de sistemas
                \item Tipos abstractos de datos para la gestión de la memoria
                en proyectos modernos: \textit{smart pointers}
            \end{itemize}
        }
    \end{SubjectTabular}

    \caption{Las dos asignaturas de programación de primer curso}
    \label{spr:programacion-i-ii}
\end{SubjectProposal}

\begin{SubjectProposal}
    \begin{SubjectTabular}
        \SubjectName{Manejo de Shell y Herramientas Básicas de Programación}
        \Semester{[1-2]} \Ects{6}
        \SubjectType{Formación Básica}
        \SubjectArea{Fundamentos de la Informática}
        \SubjectDescription{
            \begin{itemize}[leftmargin=*]
                \item Introducción a Linux
                \item Manejo del intérprete de órdenes
                \item Herramientas de desarrollo:
                editor, compilador, enlazador, depurador,
                sistemas de control de versiones y
                herramientas para la compilación de proyectos.
                \item Jerarquía de traducción y generación de código.
                \item Automatización de tareas mediante programación de scripts.
                \item Herramientas de documentación
            \end{itemize} \\
            \textbf{Seminarios}           &
            \begin{itemize}[leftmargin=*]
                \item Instalación de Linux y uso básico de la terminal
                \item Introducción a LaTeX
            \end{itemize}
        }
    \end{SubjectTabular}
    \caption{Asignatura básica de introducción a las herramientas informáticas.}
    \label{spr:manejo-shell-herramientas}
\end{SubjectProposal}

\begin{SubjectProposal}
    \begin{SubjectTabular}
        \SubjectName{Fundamentos de Computadores}
        \Semester{2} \Ects{6}
        \SubjectType{Formación Básica}
        \SubjectArea{Fundamentos de}
        \SubjectDescription{
            \begin{itemize}[leftmargin=*]
                \item Introducción:
                evolución y desarrollo histórico de los computadores,
                esquema de funcionamiento de un ordenador
                (arquitectura von Neumann)
                y arquitectura básica de un Computador.
                \item Representación de la información:
                datos de tipo entero, real y carácter,
                operaciones con los datos básicos,
                nociones básicas sobre representación de datos complejos.
                \item Sistemas digitales:
                Álgebra de Boole, circuitos combinacionales,
                circuitos secuenciales y circuitos aritmético-lógicos.
            \end{itemize} \\
            \textbf{Seminarios} & Seminario de introducción a las redes
        }
    \end{SubjectTabular}
    \caption{Asignatura de fundamentos de la informática}
    \label{spr:fundamentos-de-computadores}
\end{SubjectProposal}

\begin{SubjectProposal}
    \begin{SubjectTabular}
        \SubjectName{Álgebra Lineal}
        \Semester{1} \Ects{6}
        \SubjectType{Formación Básica}
        \SubjectArea{Fundamentos matemáticos}
        \SubjectDescription{
            \begin{itemize}[leftmargin=*]
                \item Matrices y sistemas de ecuaciones lineales
                \item Espacios vectoriales y aplicaciones lineales
                \item Sistemas de ecuaciones
                \item Diagonalización
                \item Algoritmos básicos de álgebra lineal
                \begin{itemize}
                    \item Resolución de sistemas de ecuaciones módulo $2$
                    \item Exponenciación rápida de matrices
                \end{itemize}
            \end{itemize}
        }
    \end{SubjectTabular}

    \begin{SubjectTabular}
        \SubjectName{Cálculo}
        \Semester{1} \Ects{6}
        \SubjectType{Formación Básica}
        \SubjectArea{Fundamentos matemáticos}
        \SubjectDescription{
            \begin{itemize}[leftmargin=*]
                \item Sucesiones y definición de límite de una sucesión
                \item Definición de límite de una función. Continuidad
                \item Derivabilidad e Integración
                \item Teorema de Bolzano y completitud de R
                \item Fórmula de Taylor
            \end{itemize}
        }
    \end{SubjectTabular}

    \begin{SubjectTabular}
        \SubjectName{Estadística}
        \Semester{2} \Ects{6}
        \SubjectType{Formación Básica}
        \SubjectArea{Fundamentos matemáticos}
        \SubjectDescription{
            \begin{itemize}[leftmargin=*]
                \item Introducción a la combinatoria
                \item Espacios de probabilidad discretos y
                variables aleatorias discretas.
                \item Esperanza matemática
                \item Espacios de probabilidad continuos y
                variables aleatorias continuas
                \item Estadística descriptiva
                \item Introducción a la inferencia estadística
            \end{itemize} \\
            \textbf{Seminarios}           &
            Análisis de complejidad en un treap y/o
            Algoritmos sobre cadenas con hashes
        }
    \end{SubjectTabular}

    \caption{Las tres asignaturas de fundamentos matemáticos de primer curso}
    \label{spr:first-course-mathematics}
\end{SubjectProposal}

\begin{SubjectProposal}
    \begin{SubjectTabular}
        \SubjectName{Gestión de Organizaciones y Habilidades Profesionales}
        \Semester{1} \Ects{6}
        \SubjectType{Formación Básica}
    \end{SubjectTabular}
    \caption{...} %TODO
    \label{spr:gohp}
\end{SubjectProposal}

\begin{SubjectProposal}
    \begin{SubjectTabular}
        \SubjectName{Fundamentos en electricidad y electrónica}
        \Semester{2} \Ects{6}
        \SubjectType{Formación Básica}
        \SubjectArea{Fundamentos Físicos}
        \SubjectDescription{
            \begin{itemize}[leftmargin=*]
                \item Fenómenos eléctricos
                \item Circuitos de corriente continua
                \item Fenómenos magnéticos y electromagnéticos
                \item Circuitos RC y RL
                \item Correinte alterrna
                \item Diodos y transistores
                \item Amplificadores operacionales
            \end{itemize}
        }
    \end{SubjectTabular}
    \caption{Asignatura de física}
    \label{spr:fisica}
\end{SubjectProposal}



% TODO 4 abril
\section{Segundo curso}

% TODO 4 abril
\section{Tercer curso}

\section{Menciones}

