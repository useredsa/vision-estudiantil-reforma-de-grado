\chapter{Introducción}

Este documento se enmarca dentro del trabajo de
la Comisión de Reforma del Grado en Ingeniería Informática.
Como parte del análisis de contexto,
la comisión propuso formar varios equipos de trabajo.
El trabajo encargado al Área Estudiantil fue elaborar un informe
que recogiese la perspectiva de los estudiantes,
concediéndonos libertad para elegir el contenido y el enfoque.
En respuesta a la petición,
el Área Estudiantil ha decidido, como prueba, elaborar una propuesta de reforma.

Conscientes del trabajo que supone elaborar dicha propuesta
y, a sabiendas de que la comisión llevará a cabo un procedimiento
que puede durar entre uno y dos años
y que constará de varias iteraciones en las que
se irá modificando otra propuesta común,
hemos tomado la decisión de elaborar este documento porque
vemos necesario presentar algo más que comentarios generales.
Nuestra intención es entrar en más detalle,
desglosar el futuro grado en cursos y
hablar sobre el progreso de los estudiantes cada año y
sobre las modificaciones al currículum.

No obstante, es evidente que como estudiantes
no estamos capacitados para elaborar una reforma de grado.
De hecho, la ``reforma'' que proponemos,
que se redacta en el \cref{chap:proposal},
no está escrita como lo estaría una propuesta real.
Por si fuese poco, realizamos este trabajo sin conocer todavía
los resultados del análisis del resto de subcomisiones.

En esencia, es importante valorar el informe como
un documento no oficial y dedicado al resto de la comisión.
Una visión inicial que los estudiantes queremos compartir
y sobre la que seguiremos trabajando.
Una visión que queremos integrar con la de los docentes.

\section{Composición del Área Estudiantil}

Los estudiantes constan de cuatro representantes en la comisión.
Por orden alfabético:
\begin{itemize}
    \item Emilio Domínguez Sánchez
    \item Eduardo Pérez Martínez
    \item Antonio José Ponce Zambudio
    \item José Manuel Ruiz Ródenas
\end{itemize}

\section{Estructura del documento}

El resto del informe se divide en tres bloques.
Cada capítulo cubre un bloque y provee los detalles suficientes para
ser comprendido de forma independiente.
En concreto, la estructura es la siguiente:

El \emph{\cref{chap:analysis}} identifica los problemas del grado actual,
basándose en
el Informe de la Comisión de Análisis del Grado en Ingeniería Informática,
y acompañándolo de las opiniones de los estudiantes que están cursando el grado.
En él también identificamos
cuáles de esos problemas se pueden solucionar con esta reforma y
cuáles recaen, en última instancia,
en manos del docente que imparta la asignatura.

El \emph{\cref{chap:approach}} describe el enfoque que el Área Estudiantil
quiere darle al nuevo grado.
Empezamos hablando de aspectos generales y después
hacemos un desglose de los contenidos por cursos,
centrándonos mayoritariamente en los resultados del aprendizaje anualmente.

El \emph{\cref{chap:proposal}} presenta una posible estructura
para el nuevo título,
siguiendo el enfoque del capítulo anterior.
Al mismo tiempo, comentamos los bloques de contenidos
propuestos por el resto de áreas
y los incluimos en la propuesta.
